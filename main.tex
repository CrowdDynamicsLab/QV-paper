% \documentclass[format=acmsmall, natbib=false, review=false, authordraft=false, anonymous=true, screen=true]{acmart}
\documentclass[format=acmsmall, natbib=false, anonymous=true]{acmart}

\usepackage{enumitem}
\usepackage{graphicx}  % another package that works for figures
\usepackage{lscape}
\usepackage{wrapfig}
\usepackage{booktabs} % For formal tables
\usepackage{cleveref} % for better references
\usepackage{subfig,caption}
\usepackage[english]{babel}% Recommended
\usepackage{csquotes}% Recommended
\usepackage[abbreviate=true, dateabbrev=true, natbib=true, isbn=false, doi=false, eprint=false, urldate=comp, url=false, maxbibnames=9, maxcitenames=2,  backref=false, backend=biber, style=ACM-Reference-Format]{biblatex}

\RequirePackage{rotating}

% % using biblatex
\let\citename\relax
% \RequirePackage[abbreviate=true, dateabbrev=true, natbib=true, isbn=false, doi=false, eprint=false, urldate=comp, url=false, maxbibnames=9, maxcitenames=2,  backref=false, backend=bibtex, style=ACM-Reference-Format,language=american]{biblatex}

\addbibresource{reference.bib}
\renewcommand{\bibfont}{\Small}

%% author color
\newcommand{\hs}[1]{\textcolor{red}{#1}}
\newcommand{\tc}[1]{\textcolor{blue}{#1}}
\newcommand{\lwt}[1]{\textcolor{purple}{#1}}

%% other packages
\usepackage[ruled]{algorithm2e} % For algorithms
\renewcommand{\algorithmcfname}{ALGORITHM}
\SetAlFnt{\small}
\SetAlCapFnt{\small}
\SetAlCapNameFnt{\small}
\SetAlCapHSkip{0pt}
\IncMargin{-\parindent}

%% Rights management information.  This information is sent to you
%% when you complete the rights form.  These commands have SAMPLE
%% values in them; it is your responsibility as an author to replace
%% the commands and values with those provided to you when you
%% complete the rights form.
\setcopyright{acmcopyright}
\copyrightyear{2020}
\acmYear{2020}
\acmDOI{10.1145/1122445.1122456}

%% These commands are for a PROCEEDINGS abstract or paper.
\acmConference[CSCW '20]{CSCW '20: The 23rd ACM Conference on Computer-Supported Cooperative Work and Social Computing}{Oct 17 -- 21, 2020}{Virtual}
\acmBooktitle{CSCW '20: The 23rd ACM Conference on Computer-Supported Cooperative Work and Social Computing,
Oct 17 -- 21, 2020, Virtual}
\acmPrice{15.00}
\acmISBN{978-1-4503-9999-9/18/06}


%%
%% Submission ID.
%% Use this when submitting an article to a sponsored event. You'll
%% receive a unique submission ID from the organizers
%% of the event, and this ID should be used as the parameter to this command.
\acmSubmissionID{123-A56-BU3}

%%
%% The majority of ACM publications use numbered citations and
%% references.  The command \citestyle{authoryear} switches to the
%% "author year" style.
%%
%% If you are preparing content for an event
%% sponsored by ACM SIGGRAPH, you must use the "author year" style of
%% citations and references.
%% Uncommenting
%% the next command will enable that style.
%%\citestyle{acmauthoryear}

%%
%% end of the preamble, start of the body of the document source.
\begin{document}

%%
%% The "title" command has an optional parameter,
%% allowing the author to define a "short title" to be used in page headers.
\title[QV vs Likert]{``\textellipsis I can show what I really like.'': 
Comparing Quadratic Voting with Likert Surveys at aligning respondents' preferences}


%%
%% The "author" command and its associated commands are used to define
%% the authors and their affiliations.
%% Of note is the shared affiliation of the first two authors, and the
%% "authornote" and "authornotemark" commands
%% used to denote shared contribution to the research.
\author{Ti-Chung Cheng}
\authornote{Both authors contributed equally to this research.}
\email{tcheng10@illinois.edu}
\affiliation{%
  \institution{University of Illinois at Urbana-Champaign}
}

\author{Wenting Li}
\authornotemark[1]
\email{wenting7@illinois.edu}
\affiliation{%
  \institution{University of Illinois at Urbana-Champaign}
}

\author{Yi-Hong Chou}
\email{hank0982@link.cuhk.edu.hk}
\affiliation{%
  \institution{Independent Researcher}
}

\author{Karrie Karahalios}
\email{kkarahal@illinois.edu}
\affiliation{%
  \institution{University of Illinois at Urbana-Champaign}
}

\author{Hari Sundaram}
\email{hs1@illinois.edu}
\affiliation{%
  \institution{University of Illinois at Urbana-Champaign}
}

% %%
% %% By default, the full list of authors will be used in the page
% %% headers. Often, this list is too long, and will overlap
% %% other information printed in the page headers. This command allows
% %% the author to define a more concise list
% %% of authors' names for this purpose.
\renewcommand{\shortauthors}{Ti-Chung Cheng and Wenting Li, et al.}

%%
%% The abstract is a short summary of the work to be presented in the
%% article.
\begin{abstract}
  A clear and well-documented document is presented as an
  article formatted for publication by ACM in a conference proceedings
  or journal publication. Based on the ``acmart'' document class, this
  article presents and explains many of the common variations, as well
  as many of the formatting elements an author may use in the
  preparation of the documentation of their work.
\end{abstract}

%%
%% The code below is generated by the tool at http://dl.acm.org/ccs.cfm.
%% Please copy and paste the code instead of the example below.
%%
\begin{CCSXML}
<ccs2012>
   <concept>
       <concept_id>10003120.10003130.10011762</concept_id>
       <concept_desc>Human-centered computing~Empirical studies in collaborative and social computing</concept_desc>
       <concept_significance>500</concept_significance>
       </concept>
   <concept>
       <concept_id>10003120.10003130.10003134</concept_id>
       <concept_desc>Human-centered computing~Collaborative and social computing design and evaluation methods</concept_desc>
       <concept_significance>500</concept_significance>
       </concept>
   <concept>
       <concept_id>10003120.10003121.10003122</concept_id>
       <concept_desc>Human-centered computing~HCI design and evaluation methods</concept_desc>
       <concept_significance>300</concept_significance>
       </concept>
 </ccs2012>
\end{CCSXML}

\ccsdesc[500]{Human-centered computing~Empirical studies in collaborative and social computing}
\ccsdesc[500]{Human-centered computing~Collaborative and social computing design and evaluation methods}
\ccsdesc[300]{Human-centered computing~HCI design and evaluation methods}

%%
%% Keywords. The author(s) should pick words that accurately describe
%% the work being presented. Separate the keywords with commas.
\keywords{Quadratic Voting, Likert Survey, Empirical Studies, Collaborative Decision Making}


%%
%% This command processes the author and affiliation and title
%% information and builds the first part of the formatted document.
\maketitle
\section{Introduction}
Likert scale survey 
is one of the most widely used methods
to obtain the participant's opinion
in the realm of human-computer interaction.
Survey participants would express
a rating across a series of measurements ---
\textit{Very agree to very disagree} or
\textit{On a scale of 1 to 5} ---
for a listed statement.
Very often, 
these opinions help
researchers or decision-makers
uncover consenses 
across a group of people.

However, there had been findings of
how researchers can 
easily misuse Likert scale surveys
either applying incorrect analysis methods 
\cite{bishop2015use}
or misinterpreting the analysis results
\cite{jamieson2004likert, pell2005use}
leading to questionable findings.
In addition, 
many research papers
do not explain the rational
behind the use of 
Likert scale surveys.
In a community that adopted Likert scale surveys
almost as the defacto standard,
we ask a fundamental question: 
``Is Likert-scale survey the ideal method
to measure collective attitudes for decision making?''

We begin by exploring one type of question
in collective decision making that aims To
elicit user preferences among $K$ options.
Research agencies, industry labs or independent researchers
often want to understand how to better allocate resources.
For example, 
ordinal scale polls were designed
to understand public opinions
on government policy \cite{pew}
because there is limited funding.
Companies deploy online surveys 
to understand how product users 
feel about the features and services
that needs further improvements
because companies have limited time 
to develop the next release.
Physical surveys can be found 
in shopping centers 
to collect an individual's experiences
for products on the shelf
because there are limited shelves.
All these examples demonstrated
how surveys are often tied to 
making decisions 
by gathering consensus
from surveying individual's attitudes.

In this study, 
we look at an alternative method
called Quadratic Voting (QV).
Published in 2015,
\textcite{posner2018radical}
proposed Quadratic voting
as a voting mechanism 
with approximate Pareto efficiency.
Under this voting mechanism,
voters were initially given 
a fixed amount of voice credits (VC).
With the credits, 
individuals can purchase 
any number of votes to support any of the statements
listed on the ballot.
However, the cost of each vote 
increases quadratically 
when voted toward the same option.
The authors proved that this 
mechanism is more efficient 
at making a collective decision 
because it minimizes welfare loss.
Since 2015, a few studies
compared Likert scaled surveys with QV 
empirically and theoretically
\cite{quarfoot2017quadratic, naylor2017first}.
\textcite{cavaille2018towards} argues that 
QV outperforms Likert-scale surveys 
among a set of political and economic issues.
Despite these findings,
we are not aware of related works that
compare Likert scale surveys and QV
with participants' underlying true preferences.
Therefore, it is unclear whether or not
and in what degree
does QV results align with participants' behaviors.
In addition, 
no current work, to the best of our knowledge,
deployed QV in the area of HCI.

To be more specific, 
we ask the following research questions:
\begin{enumerate}[label={},leftmargin=\parindent]
    \item RQ1. How does results from 
               QV, Likert scaled survey
               align with people's behavior 
               when surveying societal issues?
    \item RQ2. How does results from 
               QV, Likert-scale 
               align with people's behavior 
               when placed in an HCI context?
    \item RQ3. How do different amounts of
               voice credits impact results of QV empirically?
    \item RQ4. What are some qualitative insights that can be observed
               when participants vote under QV?
\end{enumerate}
To answer these research questions,
we designed two experiments.
The first experiment,
designed to answer RQ1 and RQ3,
is a between-subject study
where participants express their attitudes
among a set of societal causes using 
QV and Likert-scaled surveys
and then donate
to organizations relevant to these organizations.
The second experiment 
created an HCI study environment,
aimed to answer RQ2,
where participants were asked about 
opinions among different video elements
and their opinions using QV and Likert-scaled surveys.
Our results showed that both experiments support
QV in providing a clean and efficient way
compared to Likert scale surveys
at eliciting participant's true preferences.

\textbf{Contributions}
Our work made several contributions to the research community. 
First, we proved empirically 
the use of QV outperforms Likert scale survey
when conducting ``choosing one in $K$'' experiments.
Second, we showed that the usability of QV
is transferrable from a generic domain to HCI.
Third, we designed a bayesian model 
that facilitates the comparison
of Likert scale surveys, QV, and behaviors.
Fourth, we developed an online experiment
to mimic real-life HCI-related decision making.
And finally, we provided the source code of our easy to deploy, 
interactive web platform for QV to the community.

\textbf{Design Implication}
TODO. Talk about interface, future work and insights.
\section{Related Work} \label{related_works}

\subsection{Limitations in Likert scale surveys}
The Likert scale is an intensity scale used to elicit participants' level of agreement, satisfaction, {\change{and perceived}} importance~\cite{likert1932technique}. Invented by psychologist Rensis Likert in 1932, the initial design of {\change{the}} Likert scale aimed to identify clusters of opinions within a crowd~\cite{joshi2015likert}. The initial Likert scale survey was a 5-point scale\footnote{The original design used: Strongly Approve (1), Approve (2), Undecided (3), Disapprove (4), Strongly Disapprove (5) as the five scales.}, and subsequent researchers 3, 7, 11, or 12-point Likert scale variations~\cite{garland2008computer,finstad2010}. Some researchers developed alternative forms of Likert scale interfaces such as slider scales~\cite{roster2015exploring} or phrase completions~\cite{hodge2003phrase} to improve usability of the traditional Likert scale survey. 

Likert scale surveys have grown in popularity since they are easy to understand and administer across domains. However, as a ratings approach, Likert scale has its limitations. The primary limitation is not able to accurately understand how participants' prioritize a set of options due to several response biases, a phenomenon where survey respondents' stated {\change{opinions do}} not align with their true opinions. 

Likert scale surveys suffer from ``acquiescence bias,'' a type of response bias where participants tend to select the same level across the entire survey~\cite{alwin1985measurement, moors2016two}. To minimize this bias, researchers typically design the same ratio of positively and negatively framed options~\cite{kuru2016improving}. 

Another type of bias, ``extreme responding,'' occurs when participants only answer on the extreme ends of the Likert scale~\cite{batchelor2016extreme, furnham1986response, meisenberg2008acquiescent}, misaligned with their true preferences. An empirical study by \textcite{quarfoot2017quadratic} observed this phenomenon --- participants either expressed polarized opinions or did not express an opinion at all, making it hard for survey creators to form an optimal strategy based on the survey responses~\cite{posner2018radical}. \textcite{cavaille2018towards} provided a theoretical explanation for this bias; respondents exaggerate their opinions on purpose to influence the outcome of the survey. Researchers have developed statistical methods and suggested best practices for better question designs to mitigate such biases~\cite{glaser2008response}.

Although researchers have designed various solutions to alleviate these response biases stemmed from the \textit{mechanism} of Likert scale, in this work, we examine QV as a potential alternative that is less prone to {\change{these response}} biases.

\subsection{Quadratic Voting}
\textcite{posner2018radical} developed Quadratic Voting (QV), a collective decision-making mechanism~\cite{lalley2018quadratic} to circumvent the tyranny of the majority in traditional one-person-one-vote mechanisms, where the majority favors one option over the rest, always limiting the voice of the minority. Since QV participants are not bound to a single vote, QV does not have such a concern. Inspired by the Vickrey-Clarke-Groves mechanism~\cite{roughgarden2010algorithmic}, {\change{in QV, the marginal cost to cast an additional vote grows proportionally to the votes already cast on that option, inducing rational participants to vote proportionally to how much they care about an issue~\cite{posner2018radical}. This design is why,}} unlike many traditional voting methods, each {\change{QV}} vote comes with a {\change{quadratic}} cost. 

Here we formally define QV. Consider collecting responses from $S$ participants, where each person has access to a fixed number of voice credits $B$ to allocate across options. Then, each {\change{person can cast more than one vote (for, or against) for each option, with the proviso that the votes have a quadratic cost. Thus, casting $n_k$ votes on option $k$ would cost $c(n_k) \propto n_k^2$ in voice credits. Furthermore, the total cost in voice credits across all options cannot exceed the budget $B$. Therefore, a person casting positive or negative votes across $k$ options has to satisfy $\sum_k n_k^2 \leqslant B$, where $n_k$ is the number of votes cast on option $k$. At last, survey creators analyze the aggregated results by comparing the total number of votes from all participants across each option. }}

Several works explored the theoretical properties of QV. \textcite{lalley2018quadratic} proved theoretically that QV's total welfare loss converges as the number of respondents increases. {\change{Similarly, a recent work by \textcite{eguia2019quadratic} theoretically proved that the probability of QV aggregates  reaching a socially efficient outcome converged to one as the number of participants increased in a resource-constrained survey. ~\textcite{Lalley2018} also proved \textit{robust optimality} properties of QV, suggesting that QV may be incentive-compatible (i.e., truth-telling is the dominant strategy for the respondent). Our study examines if QV can elicit incentive-compatible results}} from an empirical lens instead of a theoretical lens. We next discuss empirical studies that compared QV with Likert scale.

\subsection{Comparing QV with Likert Scale}
Since QV is a relatively new voting mechanism, we are only aware of two studies that empirically compared QV with Likert scale. Both of these studies focused on comparing the characteristics of responses from QV and Likert scale surveys.

\textcite{quarfoot2017quadratic} surveyed 4500 participants on their opinions for ten public policies; each participant completed either a Likert scale survey, a QV survey, or both.  The study found that, for the same group of participants, responses on any option from the QV survey followed a normal distribution while those from the Likert scale survey were heavily skewed or polarized into ``W-shaped'' distributions. Researchers also noticed that individuals deliberated their responses more in QV and revealed more fine-grained attitudes. Thus, the study concluded that QV provided a clearer picture of the crowd's opinions to policy-makers on polarized issues~\cite{quarfoot2017quadratic}. 

Even though the study showed that the Likert scale survey and the QV survey produced \textit{different} results, it did not compare the survey {\change{results}} to the participants' true preferences. Our study takes one step further and compares their degree of alignment with participants' true preferences. Besides, the study focused on controversial policies, such as ``same-sex marriage'', on which voters had a strong tendency to agree or disagree at the extremes. It's unclear how QV and Likert perform when survey options are less polarized, e.g., choosing one's favorite ice cream flavor. Our study analyzes this latter scenario.

In another empirical study, \textcite{naylor2017first} utilized QV for an educational research to understand students' opinions towards a list of factors that impacted their success at universities. Results showed that QV provided more insights than the Likert survey, such as distinguishing good-to-have factors from must-have ones. {\change{Again, our study}} differs from this work as we focus on {\change{comparing}} the accuracy of survey responses.

Overall, to the best of our knowledge, prior work {\change{has neither}} studied the degree of alignment between QV responses and participants' true preferences {\change{nor compared}} it with the degree of alignment between Likert responses and true preferences. Therefore, our work is the first to examine QV's ability to elicit true preferences.

\section{Methods -- Experiment 1: Choosing among Independent Options}~\label{method_exp1}

We designed a between-subject randomized controlled experiment to answer RQ1: how well do results from QV align with people's incentive-compatible preferences compared to Likert scale when the survey aims to choose among $K$ independent options of the same subject matter? The study was in the context of a public opinion polling to understand participants' preferences towards various societal causes, such as the environment, education, veteran. We focused on the topic of societal causes because public goods and resource allocation across causes is a problem relevant to every citizen. Since resources are limited in public sectors, this problem is a typical example of choosing among $K$ independent options. Each participant completed one of the two kinds of surveys on the importance of the nine societal causes and a donation task. We detail the flow of our experiment in this section.

\subsection{Participants Recruitment}
We recruited participants located in the US from Amazon Mechanical Turk (MTurk) through the CloudResearch platform \cite{litman2017turkprime}. We made our best efforts to align the participants' age and education level distribution to the 2018 United States census estimates. We randomly assigned them into two groups: the Likert group and the QV group. Participants in the Likert group received \$0.75, and those in the QV group received \$2.5 due to longer study lengths.

\subsection{Experimental Flow}
\begin{figure}[htpb]
    \centering
    \includegraphics[width=\textwidth, keepaspectratio=true]{content/image/exp1_flow.pdf}
    \caption{
        Experiment one was a between-subjects experiment. We randomly assigned participants into two groups. Participants that took the upper path were in the Likert Group, and they expressed their attitudes toward various social causes through a five-point Likert scale. QV group participants reported their attitudes through two of the three variations of QV survey, with $36$, $108$, and $324$ voice credits respectively.
    }
    \label{fig:exp1_image_flow}
\end{figure}

\Cref{fig:exp1_image_flow} summarizes the experimental procedure. The experiment consisted of four steps: 1) demographic survey, 2) Likert or QV survey, 3) distraction survey, and 4) donation task. We provide the complete experimental protocol in the supplementary materials.

% Participants filled out the demographic survey as the first step. Based on the demographics, participants completed one or more surveys, as highlighted in the figure. Participants in the Likert group filled out one Likert survey, while participants in the QV group completed two QV surveys. After that, participants filled out another survey, the distraction survey, to divert their attention before completing the final task. We ask participants to donate to a list of charities. We explain each of these steps in detail.

% \begin{enumerate}[label=\textbf{Step }\arabic{*}\textbf{:}, leftmargin=0.45in]
\begin{description}
\item [\textbf{Step 1: Demographic Survey.}] Participants joined the study under the impression that the goal of the study was to understand their opinions towards \textit{the importance of various societal causes}. After signing the consent form, participants completed a demographic survey that asked for their gender, ethnicity, age, household income level, education level, and current occupation.

\item [\textbf{Step 2: Group 1 -- Likert Survey.}] The experiment randomly assigned some of the participants to the Likert group, shown in the upper path of \Cref{fig:exp1_image_flow}. In the prompt, we explicitly told the participants that there are limited resources in the society, and people have different preferences at allocating resources to various societal causes. The survey focused on nine societal issues, including 1) pets and animals, 2) arts, culture and humanities, 3) education, 4) environment, 5) health, 6) human services, 7) international causes, 8) faith and spiritual causes, and 9) veteran\footnote{For detailed definitions of each cause, please refer to~\Cref{cause_def}}. We derived the nine societal causes from the categorization of charity groups on Amazon Smile \footnote{https://smile.amazon.com/}, a popular donation website that has accumulated over 100 million dollars of donations.

\hspace{3 mm} We asked the participants to rate each of the nine societal issues on a 5-point Likert scale: ``For each of the issues listed below, how important do you think the issue is to you and that more effort and resources should be contributed towards the issue?''. The 5-point Likert options ranged from ``Very important'' to ``Very Unimportant.'' While there are a variety of Likert scales (3-point, 5-point, 7-point, and even 11-point), we used a 5-point Likert scale since it is one of the most commonly used scale \cite{dawes2008data}.

\item \textbf{Step 2: Group 2 -- QV Survey}

The QV group took the lower path in \Cref{fig:exp1_image_flow}. We first asked participants to watch a pre-recorded tutorial video that introduced how QV works and how to use our QV interface since we did not expect participants to know about QV before taking part in the study, as opposed to Likert scale. Participants had unlimited time to interact with a demo QV interface to familiarize themselves with QV. To ensure that the participants paid attention and understood QV, they needed to correctly answer at least three of the five multiple-choice quiz questions related to QV to continue with the study.

\hspace{3 mm} Once they passed the quiz, participants encountered two of the three versions of the QV surveys at random. The three versions of QV had 36, 108, and 324 voice credits, respectively. We showed them the same prompt as in the Likert group and instructed them to answer the same question for the nine identical causes, but with QV instead. Participants cast positive votes for causes they considered important and vice versa.
% They vote in QV using these voice credits on the nine identical options presented to the Likert Group. Participants would repeat this action using a different voice credit. We show these two QV surveys as two QV icons in \Cref{fig:exp1_image_flow}.

\hspace{3 mm} To our knowledge, no prior work discussed about how to decide on the voice credit budget in QV empirically. Therefore, we designed three versions of the QV survey and divided the QV group participants into six subgroups to answer the second question within RQ1: how does the amount of voice credits in QV impact QV's ability to elicit incentive-compatible preferences? To examine how larger voice credit budgets impact people's choices, we set an exponential increase based on the number of options ($K$) on the survey ($O(K)$, $O(K^{1.5})$ to $O(K^2)$). We investigated three levels of voice credits: $K \times O$, $K^{1.5} \times O$, and $K^2 \times O$, where $K$ is the number of options in the survey and $O$ is the number of credits required to express an attitude in QV that is equivalent to the strongest attitude in a 5-point Likert survey, where ``Neutral" in Likert = 0 vote in QV and one level in Likert = one vote in QV. In this experiment, $K=9$ corresponded to the $9$ societal causes. We used a 5-point Likert survey with extreme levels at $+/-2$; hence the participant need four voice credits ($2^2=4$) to express the extreme Likert levels in QV, which translated to $O=4$. Thus, the three levels of voice credits in the experiment were $36$ (QV36), $108$ (QV108), and $324$ (QV324). In all three cases, participants could afford to express any results from Likert in the form of QV. 


\item \textbf{Step 3: Distraction Survey}
After both groups of participants completed their surveys on the nine societal causes, they answered a free-form text question about their thoughts on another set of societal issues unrelated to those presented in the previous stage, such as increasing funding for Medicaid, strengthening gun control, and tighten social media regulation. For a complete list of causes, please refer to the appendix. We intentionally designed this survey to prevent participants from directly connecting their survey responses with the upcoming donation tasks. 

\item \textbf{Step 4: Donation Task}
In the last step of the experiment, we need to design an incentive-compatible mechanism to elicit participants' truthful preferences towards the societal causes in the QV and Likert surveys. We designed a voluntary donation task with lottery-incentives, where their donation amount should reflect how much they truly care about the causes. In this task, to the best of their interest, they should donate more to organization A than organization B only if they care more about the cause of organization A. 

\hspace{3 mm} We believe voluntary donation was a suitable task to collect the participants' truthful preferences for three reasons. First, many prior work \cite{Xiao2019, benz2008people, gendall2010effect, hsieh2010pay, hsieh2016you} used donation as an indicator of participants' preferences. Second, voluntary donation is ecologically valid, a task that occurs in real-life settings. Lastly, donation does not require specialized knowledge and is simple to conduct online and at scale. We now describe how our donation mechanism worked.

\hspace{3 mm} We showed participants a list of nine charities in a randomized order with an introduction and an official website. We selected one charity for each of the nine societal causes in the QV and Likert surveys via Amazon Smile. But we chose not to show their mapping to the causes explicitly to the participants to maintain as much independence between the survey responses and the donation decisions as possible. 

\hspace{3 mm} Participants had the chance to win \$35 as a bonus through a lottery with an odds of 1 in 70. We asked them whether they would like to donate part of this bonus to any of the nine charity groups if they were the winner. They would keep the remaining amount after the donation to themselves. While participants could donate any amount to any organization as long as the total donation value did not exceed \$35, they would want to maximize their bonus, which encouraged them to be truthful about the amount to donate. To increase participants' chance of donating a non-zero total amount, the research team promised to match \$1 to each dollar they donated to an organization and execute the donation on their behalf if they won the lottery. 
% This setup meant that the donation behavior carried an underlying cost.

% Fourth, donating is a behavior that contains complimentary and homogenous choices, and each of the options is independent. The donation task is a clear example of choosing one out of $K$ in a real-life setting.
% Adult individuals regularly exercise monetary behaviors and make financial decisions daily. These characteristics of the donation task mean that participants should not have challenges completing this task. 

% To minimize the difference across groups in the study, we used the same prompt across the Likert survey, QV, and the donation task. We explicitly told the participants that there are limited resources in society, and people have different preferences at allocating resources. We asked the participants, 
% ``What societal issues need more support?'' across all our surveys.

% \end{enumerate}
\end{description}

\subsection{System Design}

\begin{figure}[htpb]
    \centering
    \includegraphics[width=0.7\textwidth, keepaspectratio=true]{content/image/qv-donation.png}
    \caption{
        Our QV interface design in the experiments. 
        We omitted the prompt in this figure.
        After multiple design iterations, 
        the final interface allows participants to vote, with real-time feedback of how the credits are allocated. 
        The progress bar design was inspired by the knapsack voting interface by \cite{goel2015knapsack}.
    }
    \label{fig:qv_donation}
\end{figure}

We designed the QV interface with the goal to reduce participants' cognitive load with visual information \cite{oviatt2006human}. \Cref{fig:qv_donation} shows the body section of the voting panel that contained a list of options to vote on. To the left of each option, participants voted using the plus and minus buttons. Buttons for an item were automatically disabled if the number of voice credits remaining did not permit an additional vote for that item. The number of blue check or yellow cross icons next to the item represented the number of ``for'' and ``against'' votes. We provided participants a bar with percentages to the right of each option, which showed the proportion of voice credits used for that option.  In the summary panel, a progress bar showed the number of voice credits the participants have and have not used out of the total budget. We floated the summary panel at the bottom of the page at anytime to ensure visibility.

For our experimental system, we used Angular.js for the front-end, Python Flask for the back-end implementation, and MongoDB Atlas for the database. The experimental system source code is publicly available \footnote{https://github.com/dummy\_url}, and so is the code for the standalone QV interface \footnote{https://github.com/dummy\_url}. 
% The QV interface repository is a stand-alone repository written in Angular and Flask.



% -- data transformation to alignment measurement
%     describe the calculation of cosine similarity angle theta
%     mention our test of checking if other factors impact total donation amount -> absolute vs. normalized donation amount
%     show histogram and other descriptive statistics of the angle data
\subsection{Analysis Method -- Opinions Alignment Metric}~\label{alignment_metric}

We break our research question on ``alignments'' into two parts. First, we ask how similar these individual survey responses are to a participant's incentive-compatible behavior. Then we ask, how does QV and Likert compare overall in terms of the degree of alignment? To answer the first question, we need a metric for alignment.

We first clarify the definition of "alignment" in our analysis. A perfect alignment between a survey response and the same participant's donation choices requires an individual to express their preferences in survey with the same relative strength as their donation amount. More formally, it is defined as the following:

\begin{quote}
    A set of survey response $\vec{v} = (v_1, v_2, \dots, v_n) \in \mathbb{R}^n$, and a set of donation amount $\vec{d} = (d_1, d_2, \dots, d_n)\in \mathbb{R}_{+}^n$, where $n$ is the number of topics or options involved in the decision, are perfectly aligned if there exists a positive constant $k>0$ that satisfies $k\vec{v} = \vec{d}$.
\end{quote}

Notice now we represent the a participant's response in Likert survey, QV, and donation each with a vector of the same length. In addition, we focus the definition of alignment on the \textit{relative} strength across opinions for two reasons. First, the results from our four types of surveys (Likert, QV36, QV108, QV324) and the donation task were not on the same scale. For example, the maximum possible number of votes on a topic in QV36 was 6, while the maximum donation amount on a topic was \$35. A relative scale maps each response onto the same space. The other reason is when any two participants donated different absolute amounts, possibly due to other factors such as income level or level of education, we want to capture how they \textit{distributed} their total donation amounts. In other words, participants might donate with the \textit{same} set of preferences across topics, but with \textit{different} levels of total amount they were willing to donate. Hence, we decided to care only about the relative strength in opinions across topics.

Next, we need a metric that measures the degree of alignment. This metric needs to be monotonic with respect to the amount of discrepancy between two preference vectors in terms of the relative strength across preferences. In addition, this metric needs to be easily interpretable. Therefore, we decided to make use of the cosine similarity metric as our alignment metric and represent the difference between the survey results and the donation amount with an angle $\theta$. It is formally defined as the following:

\begin{quote}
    The cosine similarity angle $\theta$ between a set of survey response $\vec{v} = (v_1, v_2,\dots, v_n) \in \mathbb{R}^n$, and a set of donation amount $\vec{d} = (d_1, d_2,\dots, d_n) \in \mathbb{R}_{+}^n$, where $n$ is the number of topics or options involved in the decision, is calculated via $\theta = \arccos \left ( {\frac{\langle \vec{v},  \vec{d} \rangle}{\|\vec{v}\| \|\vec{d}\|}} \right )$, $\theta \in (0, \pi)$.
\end{quote}

Cosine similarity is a commonly used similarity metric 
that measures the cosine of the angle between two non-zero vectors \cite{singhal2001modern}. Instead of reporting a value between $0$ and $2 \pi$ radians, we report the angle in degrees, allowing for a more intuitive interpretation.
%The definition of the angle of Cosine similarity fits our need perfectly. 
It is monotonic with respect to the relative orientations of the two vectors, i.e., the relative strength in opinions, and does not take into account the magnitude of the vectors, i.e., absolute vote or donation amount. Two sets of perfectly align opinions will yield a cosine similarity angle of zero, while two sets of completely opposite opinions will result in an angle of 180 degree.

% Intuitively,
% to compute the Cosine similarity angle 
% for each participant in each group,
% we first translate each response,
% urvey results from Likert, QV36, QV108, QV324, and 
% their corresponding truthful preferences 
% reflected in the donation task,
% into a vector. %mentioned above can repeat if necessary
For the Likert group, we map the ordinal responses into a vector where the result for each topic ranges from $-2$ to $2$. For each of the three QV conditions, the vector contains the number of votes of the topics as is. Then, for each individual, we computed the cosine similarity angle between the Likert or QV vector and the absolute donation amount of the same individual. 

Once we gathered these data, we moved on to the next step where we uncovered how the cosine similarity angles compared across the Likert group and the three QV variances (QV36, QV108 and QV324). We set up a Bayesian Model with these four sets of cosine similarity angle for each condition, as described in the next subsection.

\subsection{Analysis Method -- A Bayesian Approach}
\label{exp1:The Bayesian Model}


% In many survey analysis experiments, researchers relied heavily on the null hypothesis statistical tests (NHST) to determine whether a phenomenon is statistically significant or not, to support their hypothesis. This method leads to controversies in the field for a very long time. One major challenge of NHST came from its goal: rejecting the null hypothesis. Instead of answering the alternative directly, this made NHST easy to overstate the evidence against the null hypothesis \cite{david2000NHST}. In addition, some researchers \cite{kruschke2010bayesian} argued that it is easy to `p-hack' an experiment by replicating an experiment repetitively and only reporting the ones with significant results. There were heated debates upon confidence intervals, alpha values, the sample size decision, and many others when discussing related issues of NHST.

We used Bayesian analysis to compare if the distribution of cosine similarity angle in the QV group significantly differs from that in the Likert group. The core concept of Bayesian analysis is updating and reallocating the belief as one gathered more information with the use of Bayes rules. Kay et al. \cite{kay2016researcher} introduced the following benefits of using this analysis method in the HCI community. First, Bayesian shifts the conversation from ``did it work" to ``how strong is the effect." While null hypothesis statistical tests (NHST) produces a single p-value and one effect size value, a Bayesian model can provide the distribution of the effect size, making additional information available for a clear inference. Second, a Bayesian model is valid at every value of the sample size while NHST that assumes normality requires at least a sample size of 30. Lastly, a Bayesian model does not require the assumptions of normality and homogeneity of variance since the researcher can foreground all the aspects of the model. 


Now, we discuss our Bayesian formulation. There is one outcome variable: $\theta_{i \mid j}$, the cosine similarity angle between a survey response vector and a donation amount vector of each participant $i$  under each of the four experimental conditions $j$: Likert, and three QV conditions (with 36, 108 and 324 credits). In summary, we aim to fit a distribution for the mean (i.e. the expected) angle between each survey instrument and the donation amount. Then we compare how different the four distributions are. 

In a Bayesian formulation, we need to define a likelihood function to model the cosine similarity under each condition. In general, this is a parametric formulation, and consistent with~\textcite{McElreath2015}. The likelihood function represents the modeler's view of the data, and not a claim about the world. The likelihood function is often parametric and we treat each model parameter as a random variable, drawn from a distribution (its prior). Typically, these priors are ``weakly informative''---conservative priors which allow for all possible values of the parameter but chosen in a manner that promotes fast convergence.


We use a Student-t distribution to characterize the mean (i.e. the expected) angle between the survey instrument (Likert and the three QV conditions). A Student-t, unlike a Normal distribution, is heavy-tailed, in the sense that the Student-t distribution doesn't fall off as quickly as does a Normal distribution and will thus be able to better account for outliers in the data. The Student-t distribution has three parameters: the degrees of freedom ($\nu$), the experimental condition dependent mean ($\mu_j$) and scale ($\sigma_j$).  These parameters are random variables and we need to define their priors. Since our goal is to model the \textit{average} angle, the fact that the Student-t is unbounded while the angle $\theta \in [0, \pi]$ is bounded is unimportant.

\begin{align}
  \theta_{i \mid j} \sim & \mathrm{Student-t}(\nu, \mu_j, \sigma_j),   & \text{likelihood function to model donation} \label{eq:bayesian formulation} \\
  \nu \sim & 1 + exp(\lambda), & \text{degrees of freedom} \\
  \mu_j \sim & N(M_0, \sigma_0), & \text{modal angle in condition } j \\
  \sigma_j \sim & \Gamma(\alpha, \beta), & \text{scale parameter for condition } j
\end{align}
 
\Cref{eq:bayesian formulation} describes that the response $\theta_{i \mid j}$ of each group $j$ is modeled as a $\mathrm{Student-t}$ distribution with mode $\mu_j$,  scale $\sigma_j$ and with $\nu$ degrees of freedom. Next, we explain the model parameters.


\begin{description}
    \item[Degrees of Freedom:] We draw the degrees of freedom $\nu$ from a shifted exponential distribution, to ensure $\nu \geq 1$; $\nu=\infty$, corresponds to a Normal distribution assumption.
    \item[Modal contribution $\mu_j$ in each condition $j$:] The mode $\mu_j$ corresponding to each group is drawn from a Normally distributed random variables with constant mean $M_0$ and variance $\sigma_0$. 
    \item[Scale $\sigma_j$ of each condition $j$:]  The scale $\sigma_j$ of the likelihood function is drawn from a Gamma distribution $\Gamma(\alpha, \beta)$, with mode $\alpha$ and scale $\beta$; this prior on $\sigma_j$ ensures that $\sigma_j > 0$. 
    \item[Constants:] The constants $M_0, \sigma_0, \alpha, \beta$ are set so that the priors are generous but weakly informative so that despite exploring all possible values, we ensure rapid MCMC convergence.
\end{description}

We performed the Bayesian analysis using PyMC3 \cite{salvatier2016probabilistic}, 
a popular Bayesian inference framework. We used one of the common computational techniques for Bayesian inference, Markov Chain Monte Carlo (MCMC), a stochastic sampling technique. It samples the posterior distribution $P(\theta | D)$, the distribution functions of the parameters in the likelihood function given the data observations $D$. 
% We used the No-U Turn Sampler (NUTS) specifically 
% in our analysis. 


%% Old draft text
% % Experiment overview
% We designed the first experiment as a between within-subject study
% consisting of Likert survey, QV, and a donation task,
% to answer research questions one and three. 
% Participants were recruited from Amazon Mechanical Turk (MTurk).
% The Likert Group received \$0.75 
% while the QV and Likert Group received \$1.75 
% when participants completed the study.
% In this section, 
% we detailed our experiment design.

% The goal of a between within-subject study is to
% understand how within-subject does a survey tool
% aligns with one's true preference.
% To compare QV and Likert, 
% these results, 
% the difference between the survey tool and one's preference, 
% were then compared across groups of participants.

% %The donation task. Why?
% In order to figure out 
% how QV and Likert surveys 
% represent an individual's true preferences,
% we ask participants to complete a survey
% and donate to a charity.
% Our goal is to see 
% whether a participant's donation behavior is similar to 
% the attitudes they stated in the survey.
% There are multiple reasons
% why we designed a donation task.
% First, donation tasks are easily relatable
% given that they occur in real life and
% and monetary behaviors are direct and imaginable.
% Second, donations are simple to conduct,
% even on a large scale.
% Third, donation tasks appeared in many experiments 
% \cite{Xiao2019, benz2008people, gendall2010effect} 
% as an effective indicator of participants' behavior.
% Fourth, donating is a behavior containing
% complimentary and homogenous choices and
% where each of the options is independent.
% It is a clear example of choosing one out of $K$ in real life.

% \begin{figure}[htpb]
%     \centering
%     \includegraphics[width=\textwidth, keepaspectratio=true]{content/image/exp1_flow.pdf}
%     \caption{
%         Experiment one conducted between and within subjects. Participants were divided into two groups. Participants that took the upper path are the Likert Group, who expressed attitudes of various social causes through a five-point Likert Survey. The alternative is the QV group who replied attitudes through two QV surveys, each with a different combination among the three possible votes: $36$, $108$, or $324$ credits.
%     }
%     \Description[Image describing the flow for experiment 1]{Image describing the flow for experiment 1}
%     \label{fig:exp1_image_flow}
% \end{figure}

% At a high level, 
% we summarize the experiment flow 
% in Graph \ref{fig:exp1_image_flow}.
% The experiment consisted of four steps:
% To begin the experiment, 
% participants filled out the demographic survey.
% Based on the demographics,
% participants completed one form of opinion collection,
% highlighted as the yellow box
% in Graph \ref{fig:exp1_image_flow}.
% After that, 
% participants filled out another survey,
% the distraction survey,
% to divert their attention before they complete the final task.
% The final task asked participants to donate.
% Now we explain each section in detail.

% Before starting the experiment,
% participants were told that 
% this study aims to understand their opinions 
% toward social causes and will be asked to complete a donation task.
% During the demographic survey, 
% we collected the participant's gender, ethnicity, age range, household income level, 
% education level, and current occupation.
% Based on the age and education level,
% we divided participants into seven groups
% and made sure each group contained the same distribution
% as the US 2019 census.
% These seven groups can be further categorized as
% the Likert Group (Group 1) and the QV Group (Group 2 to 7).
% The Likert Group, shown as the upper path in the shaded area of Graph \ref{fig:exp1_image_flow}, 
% revealed their opinion using a Likert survey.
% The QV Group, shown as the lower path in the shaded area of Graph \ref{fig:exp1_image_flow}, 
% revealed their opinions by completing two QVs, each with different numbers of voice credit.
% We divided the QV Group into six subgroups
% to answer research question three, 
% which is whether the number of voice credits impacts the outcome.
% These two voice credits that participants experience 
% are drawn from three possible values: $N \times O$, $N^{1.5} \times O$, $N^2 \times O$, 
% where $N$ is the number of options in the survey, 
% and $O$ is the number of levels, 
% excluding neutral on the Likert survey. %not sure if this is clear
% In our case, with nine options ($N=9$) and
% used a five-point Likert survey ($O=4$), 
% the three values would be $36$, $108$, and $324$.
% With these three possible values, 
% we choose two for each of the six subgroups.

% In the Likert group, 
% the survey looks identical to a typical five-point Likert survey.
% We assume participants have prior knowledge in Likert surveys.
% Participants were presented with the nine societal causes, 
% and were asked the importance each of these causes: 
% With options ranging from ``Very important'' to ``Very Unimportant.''

% In the QV group, 
% participants were asked to watch 
% a prerecorded tutorial video of QV's concept 
% and how to operate the QV interface.
% Participants are granted unlimited time 
% to interact with a demo QV interface. 
% This process is demonstrated as 
% ``tutorial on quadratic voting'' 
% in Graph \ref{fig:exp1_image_flow}.
% To ensure that participants paid attention to the video and understood QV, 
% they were asked to answer at least three of the five multiple-choice questions 
% correctly to continue with the survey.
% Once participants passed the quiz, 
% participants will be given voice credits of either 36, 108, or 324.
% They will vote in QV using these voice credits 
% with the nine options identical to those in the Likert Group.
% Participants would repeat this action using a different set of voice credits.
% These two QVs are shown as two QV icons in Graph \ref{fig:exp1_image_flow}.

% After both groups of participants completed their surveys in the opinion collection stage, 
% they finish a short answer question
% that allowed them to express their thoughts 
% related to another set of societal issues.
% These societal issues are unrelated in the previous stage,
% and are designed to distract participants.
% We do not want participants to connect their survey responses
% to interfere with their behaviors during the donation task.

% Finally, 
% we ask participants to perform a donation in the final stage.
% This task presented nine organizations,
% each referring to one of the nine societal causes
% that we listed during the opinion collection phase.
% Participants can donate 
% any amount to any of the listed organizations
% without exceeding a total of 35 dollars.
% To ensure incentive compatibility, 
% participants do not donate imaginatively.
% Participants are aware that every one in 70 participants would win 35 US dollars.
% Assuming winning the 35 US dollars, 
% the participants were asked 
% if they would want to donate some money 
% to any of the nine charity groups.
% Participants are also aware that 
% they keep the remaining amount of undonated money 
% if they win the lottery.
% Further, participants are aware that 
% the research team will match one dollar to each one dollar 
% they donated to an organization.
% \tc{We need to justify why we matched}
% This setup means the donation carried an underlying cost.

% To minimize the difference across groups in the study, 
% we use the same prompt across Likert survey, QV, and the donation task.
% We explicitly tell the participants that 
% there are limited resources in the society, 
% and people have different preferences 
% in how resources should be allocated and ask the participants, 
% ``What societal issues need more support?''

% To ensure that the nine societal causes 
% covered a broad spectrum of categories.
% We used the categorization of charity groups on Amazon Smile, 
% a popular donation website that has accumulated over 100 million dollars of donations, 
% as our topics of the societal causes.
% The categories include:
% \begin{enumerate}[label={},leftmargin=\parindent]
%     \item (1) Pets and Animals
%     \item (2) Arts, Culture, and Humanities
%     \item (3) Education
%     \item (4) Environment
%     \item (5) Health
%     \item (6) Human Services
%     \item (7) International
%     \item (8) Faith and Spiritual
%     \item (9) Veteran
% \end{enumerate}
% Within each of these categories, 
% we select one charity organization from Amazon Smile 
% as the representation of the subject matter used in the donation task.

% \subsection{System Design}
% We use Python Flask for the back-end, Angular for front-end, 
% and MongoDB for database storage to construct the voting system. 
% The experiment source code is publicly available \footnote{Not yet public}, 
% and so is the QV interface as a stand-alone repository \footnote{https://github.com/hank0982/QV-app}.

% \begin{figure}[htpb]
%     \centering
%     \includegraphics[width=0.7\textwidth, keepaspectratio=true]{content/image/qv-donation.png}
%     \caption{
%         The QV voting interface used across both experiments. 
%         We omit the prompt in this figure.
%         After mutiple iterations (details in the Appendeix), 
%         the interface allows participants to vote, 
%         with real time feedback of how the votes allocats. 
%         The progress bar implementation 
%         were inspired by knapsack voting interface by \textcite{goel2015knapsack}.
%     }
%     \label{fig:qv_donation}
% \end{figure}

% The QV interface, is shown in Figure \ref{fig:qv_donation}.
% The body section is the voting panel
% that contained all options to vote for.
% To the left of each option, 
% participants vote using the plus and minus buttons.
% Buttons are disabled 
% if the number of voice credits 
% does not permit the next vote.
% A bar on the right of the option 
% shows the proportion of voice credits 
% used to that option with text associated with the visual.
% The different colors and the icons 
% to the right of each option 
% exhibits the number of for or against 
% that currently devoted to an option.
% The summary panel always 
% floats at the bottom of the page 
% to ensure visibility.
% A progress bar shows the number of voice credits 
% that the participants have and had not used.\par
\subsection{Experiment 1 Results} \label{results-1}
\subsubsection{Quantitative Analysis Results}

% -- raw data
%     describe the dataset, total # of participants in each group before and after dataset cleaning, demographics of each group;
    
%     donation descriptive statistics: perc of non-zero donation, total donation amount comparison across groups (discuss our test of checking if other factors impact total donation amount -> absolute vs. normalized donation amount), donation distribution across topics between groups
    
%     QV & Likert votes descriptive statistics: votes distribution per topic across groups, budget usage distribution across QV groups
    
% -- data transformation to alignment measurement
%     describe the calculation of cosine similarity angle theta
%     show histogram and other descriptive statistics of the angel data
    
% -- Bayesian formulation
%     why Bayesian
%     the type of analysis question
%     choice of the likelihood function
%     choice of prior distributions
    
% -- Results analysis
%     Tools: PyMC3, MCMC, NUTS
%     Describe fitted values & convergence (trace plots)
%     Describe effect size analysis for comparing Likert and QV
%     Describe effect size analysis for comparing across QVs
    
\subsubsection{Qualitative Analysis Results}
We ask participants to provide a freeform text response on the reason why they made the choices they made
when participants filled out the Likert scale survey or QV survey,
Of all surveys ($N=394$) across both groups, most participants filled out the surveys ($N=331$) based on what they think are the most important issues to them. %84 percent
Besides, a small portion of participants ($N=30$) used their instincts when replying to the survey.
Some participants either think that every aspect is important ($N=7$) or that resources should be equally distributed ($N=7$).

For the participants that said they reply based on what they think is most important to them, 
participants usually perform an ``electing'' over a handful of issues first and then indicate their preference.
% limitation at showing intensity @ Likert
However, we see a few instances in the Likert group where participants would claim one to two options as the most critical causes but electing three or more ``very important''.
For example, P09a47 mentioned, ``I think the environment education and healthcare should be our top priorities right now. Other issues are also important but not as much so.''
while filling out Education, Environment, Health, and Human Services as ``Very important.'' 
Pd80fc mentioned, ``[I] think health and the environment are important'' while putting Pets and Animals, Arts, Culture, Humanities, Environment, and Veteran. as ``Very important.'' 
Though it is unclear why there is this discrepancy, one participant (P9b3ae
) explicitly mentioned ``[\textellipsis] I would answer otherwise, if there were other options, such as not much, or a little bit.''
Similar issues were not present in the QV Group responses.
Participants were able to express more fine-grain preferences.
P1fee1 mentioned, ``I think health and human service are important and beneficial for society.'' while voting six votes for health and five votes for human services. This indicates that despite being ``important'', there is still a difference in weight and shadowed the limitation of Likert scaled survey, where people can be limited by the options they were given.\par

Despite only occurring once (P1d659), we think the term ``voting'' might motivate participants to consider how a collective decision was made.
This participant mentioned ``[\textellipsis] I thought a lot of people would probably support faith and spiritual ecatagory  so I watned to try to counterbalance that by voting against it.'' The participants are willing to pay the cost by devoting fewer votes for issues they care and vote against specific ballots. This behavior is not possible in a Likert scaled survey.\par

To understand how participants in QV voted, we also analyze how participant's responses changed when the number of voice credits changed. 
If the number of voice credits increased, as expected, some participants uniformly increased the number of votes across all nine options, stating that they try to be fair. Also, logically, other participants would devote the additional voice credits to the items of their likes or dislikes. 
It is, however, fascinating to identify how additional credits pushed participants to express more fine-grain preferences.
For example, some participants that had a drastic increase of voice credits, from 36 to 324 voice credits, expressed devoting some options that originally had zero votes. 
P2d9da stated, ``Because now that I have a lot more credits, I felt that I could vote on more issues that mean something to me.'' The participant initially only voted for Environment; however, with 324 votes, the participants voted for all but Faith and Spiritual. This supports our quantitative finding that a limited amount of voice credits suppressed the performance of QV. Participants also reported being freer and submitted more fine-grain opinions. As one participant (P54f23) responded: ``The greater voice quantity allowed me to vary the differences in choices'' more and similarly Pcc4aa reported, ``with more credits i can show what i really like.''

On the contrary, participants are forced to downsize their preferences if credits decreased. Many participants voiced their need to make tradeoffs. P9e5e6 said, ``I think I covered the bare basics.'' and Pe37f2 said, ``Less to go around, so had to knuckle down and allocate the most to what I think is most important.'' Again, this means that it is crucial to have enough points if we want to reflect participant's preferences and they're intensity accurately.

One thing to notice is participants scaled their preferences based on the number of total voice credits. Even though voice credits are one single unit and do not carry any weight, when total voice credits increase, the value of each voice credit devalued, vice versa. P24194 stated, ``I had fewer credits so each vote seemed more expensive.''

This set of qualitative analysis support the quantitative finding that the number of voice credits impacts the performance of QV and there is a need to find the best way at deciding \textit{what} the number of voice credits should be.
\section{Experiment 2}

We designed two experiments to investigate our research questions.

The second experiment extends upon the first one with a focus
in the context of HCI survey.\par


\subsection{Methodology} \label{method-2}

\subsubsection{Experiment 2 Design}

\section{Experiment 2}
The second experiment 
extends upon the first one,
and focus in an HCI setting.
In other words, 
does the same result from experiment one
apply to another domain.
Different from political and public-opinion surveys, 
HCI surveys with eliciting one in $K$,
usually focuses on users' preference 
in interface design and user experiences.
However, this also makes measuring 
participant's true preference
much more non-trivial.
Thus, we developed a buy-back mechanism 
and observe participants' behaviors in that task
which serves as their true preferences.
This experiment also acts 
as a concrete example as to how 
QV can be incorporated in HCI.

\subsection{Choice of HCI Research Question}
Selecting a HCI research question
for us to apply QV, Likert, and 
have a task that reflects participant's behavior
is not trivial. 
At the same time, 
we do not want to invent a new experiment 
that requires complex verification.
Similar in the pretest,
we want a well explored HCI topic 
that we could rely on
to ensure ecological validity.
Most HCI research uses Likert scale surveys
to understand participants opinion
across one or more devices, design or interfaces.
One could view this as one form of eliciting one
out of $K$. 
However, reproducing one of these experiences
can be costly and difficult.
Therefore, we turn to the other type of experiment,
where UX/UI researchers aimed to prioritize
features and elements that their customers care about.
This type of surveys are often find online
and as feedback forms.

Research on video and audio quality 
from the lens of HCI 
has been a relatively mature.
Contributions has been made to fields like
multi-media conferencing \cite{watson1996evaluating}, 
video-audio perception \cite{chen2006cognitive, molnar2015assessing}
and more specifically trade-offs 
between video and audio elements 
under network monetary constraints \cite{molnar2013comedy, oeldorf2012bad}.

Oeldorf-Hirsch et al. \cite{oeldorf2012bad} conducted a study, 
covering the widest range of elements 
to the best of our knowledge,
to understand how users 
with bandwidth constraints 
made trade-offs between video and audio elements. 
They examined participants' attitude 
between three video bit rates, 
three video frame rates 
and two audio sampling rates 
across three types of video content.
Participants were asked to rate the overall quality, 
video quality, audio quality and enjoyment level 
on a 5-point Likert scale in each condition. 
Conclusion were drawn 
using mean and standard deviation 
of the survey results.
This is a typical study 
where the goal is to find  
one or some of the $K$ elements to choose from
when under constraint.
We follow similar experiment scenario
with emphasis on collecting participant's attitude
among a wider range of video and audio elements 
and compare how Likert scaled survey 
and QV reflects people's true perception preferences. 

Based on these related works, 
we selected five video elements to alter
in our experiments. 
These elements includes: 
(1) Audio Quality, 
(2) Video Quality,
(3) Audio Packet Loss,
(4) Video Packet Loss, and 
(5) Audio-Video Synchronization.
Details of these elements 
are discussed in the next subsection.

% \begin{figure}[htpb]
%     \centering
%     \includegraphics[width=\textwidth, keepaspectratio=true]{resources/exp_2_video.png}
%     \caption{
%         Real-time Video Element Interface
%     }
%     \label{fig:exp_2_video}
% \end{figure}

\subsection{Video Alternation Interface}
The key interface in this experiment
is the real-time video element interface
displayed in Figure \ref{fig:exp_2_video}. 
This interface showcased a weather video
with a set of controls at the bottom.
Participants can toggle any of these video elements,
and see the immediate changes to the video
on the top of the interface.
Participants can pause and play the video at anytime.

Based on prior study on the perception of 
the degradation of the 5 features, 
we designed the 4 levels to be as linear 
to user's perception as possible. 
Below are the 4 levels of the 5 elements 
listed from the worst to the best quality
reflected on the interface.

\begin{enumerate}
    \item Stability of Video Imagery \cite{claypool1999effects}: This refers to mimicking the effects of lost packets of the video. When packets are lost during transmission, the screen would freeze in the previous frame. The different levels are set with 20\%, 8\%, 4\%, and 0\% of the data lost during the entire video.
    \item Stability of Audio \cite{claypool1999effects}: This refers to mimicking the effects of lost packets of the audio. When packets are lost during transmission, the audio would drop for a certain amount of time. The different levels are set with 20\%, 8\%, 4\%, and 0\% of the data lost during the entire playback.
    \item Quality of audio \cite{oeldorf2012bad, noll1993wideband}: The quality can change with a different audio sampling rate that refers to the different file size of the audio. The different levels of audio sampling rate in the experiment is 8kHz, 11kHz, 16kHz, and 48kHz respectively.
    \item Quality of the video \cite{oeldorf2012bad, knoche2008low}: The quality of the video is alternated by changing the video resolution 210x280, 294x392, 364x486, and 420x560 but fitted into the same size of final display such that the pixel density per inch differs.
    \item Video-audio Synchronization \cite{steinmetz1996human}: The synchronization of the video and audio is altered by having the audio 1850, 1615, 1050 or 0 milliseconds ahead of the video.
\end{enumerate}

In the prompt of the interface, 
participants were told that this research 
is conducted by a video streaming company
primarily serving in-flight entertainment systems.
During flights,
data bandwidth are limited and engineers 
of the company need to know what to prioritize
to serve to the customer.
Therefore, through the survey,
the company can understand 
how customers think are important video elements
that will help them understand the video.

We believe this scenario is easy to understand 
and can be easily applied to many real life scenarios,
such as sudden drop in mobile network,
spotty WiFi connection
or really in a flight.
We also believe that participants 
have experience at least one of the five
video elements in the past
making this easy to understand and relay.


\subsection{Buyback as behaviors}
Similar to the first experiment,
we need a task to align participant's attitude
with their behavior.
We designed a task called Buyback, 
which mimics a rational customer's behavior: 
buying essential tools to complete some given task.
To the best of our knowledge,
we are the first to design such a task,
yet, it reflects many behaviors
in the real life.
The task stems on 
many subscription-based services on the market,
in which requires customers to pay additional premiums
for additional benefits.

In the first stage of the buyback,
participants were given 
a video with sub-optimal quality
that mimics worst case scenarios
if the internet bandwidth is limited.
This is realized by 
setting the video interface controllers 
to level 0.
Participants can ``enhance'' each of these elements
buy ``purchasing'' a level of that element.
For instance, 
participants can buy two levels of audio quality,
and one level of video stability.
Each of these levels costs $2$ dollars.
Participants were given a budget of \$30 
to purchase some or all of the features back.
We call this action the ``buyback actions''.

To ensure incentive-compatibility 
of the participants' buy-back actions, 
we offered to pay the participants their own remaining amount 
from the \$30 budget through a lottery.
Participants would be eligible for the lottery 
only if they correctly answered 80\% of 
five multiple choice questions 
related to the content of the video correctly.
This ensures that participants has
correctly comprehend the video.
This also set the goal for the participants,
to really consider what video elements
impact one's experience at understanding
the content of the video.

The five questions are factual questions such as, 
"What is the weather of Chicago?", 
"What is the highs and lows of San Diego", and
"Which city was not shown in the video?". 
Participants were shown five example questions 
before the buy-back task to assist their decision.
Participants can replay the video with their adjustments
while answering the questions
to ensure that participants do not require memorization.\par

% \begin{figure}[htpb]
%     \centering
%     \includegraphics[width=\textwidth, keepaspectratio=true]{resources/Exp2.png}
%     \caption{
%         Experiment 2 Flow Chart
%     }
%     \label{fig:exp2_flow}
% \end{figure}

\subsection{Experiment setup}
In this experiment,
we recruit 180 participants through MTurk.
Participants were divided into 3 groups, 
demonstrated in Figure \ref{fig:exp1_flow}.
We design the study as a between subject study.

After agreeing the consent form,
all three group of participants 
will complete a demographic survey.
This demographic survey will also
captures participant's basic information 
such as age, gender, income, ethnicity, profession and so on. 
Shown in Figure \ref{fig:exp_2_video}, 
the three groups of participants,
from top to bottom, are Likert, QV and Buyback.
Now we describe the experiment design for each group.

In the Likert group,
participants were ask to read through a page
that contains the explanation of the five video elements
listed above.
Participants will then have a chance to experience
the video interface,
to understand how different video elements
impact a video.
Participants are also required to answer the five sample questions
displayed to the participants in the buyback task.
This assures that the participants have the same goal of 
trying to understand the context of the video
and not just for pure entertainment purposes.

Once the participants have spend enough time with the video elements
as well as answering the five multiple choice questions,
they are asked to fill out a likert-scale survey,
expressing their attitude
toward the the different video elements.

The QV group follows closely with the Likert Group.
Participants in the QV group 
are first required to look at a short clip
on what QV is and answer a few questions
to make sure that they understood how QV works.
Then, similar to the Likert group, 
participants will learn about the video elements, 
experience it through the video element interface.
They would then asked to ``vote'' on how important they think
the different video elements were,
to help them comprehend the video content.
In this QV, we use 100 credits based on the optimal
results from the first experiment.
We use the same QV interface demonstrated in Figure \ref{fig:system_interface}.

Participants in the buyback group,
similar to the Likert group,
will also learn about the different video elements
and experience it in the video element interface.

They would then shown the buyback task as described earlier
in which they will decide
what they are buying
based on the demo video.
With the budget of \$30 and each feature costs \$2,
participants can buy back everything with no extra payoff
or making trade-offs between the elements they think are important.

Once their decision is made,
we will show the participants another video, 
very similar to the first video,
using the set of controls they bought
and ask them a few questions.


% In our experiment, we included a total of five video and audio element that will impact a video.
% These elements include video and audio package loss rate, 
% determining whether the audio or video stutters; 
% video resolution and audio sampling rate 
% effecting the quality of video and audio; 
% and video-audio synchronization. 
% We selected a few segment of weather broadcasting 
% from a news channel 
% as the content of our video.
% Weather broadcasts usually convey information via both visual and audio channels, 
% appeal to a wide array of audiences, 
% and do not require prior knowledge to understand.\par

% To ensure the ecological validity of the experiment, 
% we situated the comparison of different video and audio elements 
% in a hypothetical scenario in which the participant 
% is a manager of a weather reporting news station. 
% As the manager, 
% the participant was asked to rate 
% the importance of each video and audio elements 
% with the goal to maximize customer understanding of the context
% where network is of low bandwidth and that the weather broadcast 
% cannot be shown in its best quality.\par

% We designed a between-subject study
% with three groups of 60 participants.
% After the participants agreed with the consent form, 
% all three group of participant
% were presented an example weather broadcast segment
% with controls of the five video and audio elements 
% under the video shown in figure M. 
% All five elements were set to sub-optimal by default, 
% making the content near incomprehensible.
% Participants can alternate the five elements 
% in any combination, 
% to see how elements impact to the video.

% Once participants think that they had a grasp of 
% how different elements impact a video, 
% the first group of participants 
% then completed a 5-point Likert-scaled survey 
% while the second group of participants 
% completed a QV survey with K voice credits \footnote{K is decided from experiment 1}, 
% asking their opinions 
% on the importance of the 5 video and audio elements 
% in a weather broadcast 
% under a low bandwidth environment. 
% The third group of participants 
% were asked to perform a buy-back task 
% for a bad-quality advertising video.\par

% The buy-back task mimics a rational customer's behavior: 
% buying essential tools to complete some given task.
% Participants were told that 
% as the manager of the weather broadcast agency,
% they need to verify if their viewer 
% can understand the content of the video.
% Therefore, the goal of their task 
% is to correctly answer a set of multiple choice questions
% to make sure that they correctly comprehend the video.
% Given the video with sub-optimal video, 
% participants were given a budget of \$30 
% to purchase some or all of the features back.
% To ensure incentive-compatibility 
% of the participants' buy-back actions, 
% we offered to pay the participants their own remaining amount 
% from the \$30 budget through a lottery 
% under the condition that 
% they answered 80\% of the multiple choice questions correctly,
% [missing probability]
% version of a new weather broadcast video 
% adjusted by their buy-back choices. 
% These questions contained factual questions such as, 
% "What is the weather of Chicago?", 
% "What is the highs and lows of San Diego", 
% "Which of the follow cities got colder?". 
% Participants were shown three example questions 
% before the buy-back task to assist their decision.
% Participants can replay the video with their adjustments
% while answering the questions
% to ensure that participants do not require memorization.
% There will be a 5 minute timer to minimize the impact of replaying the video.
% With this design, 
% participants would try their best to make the video comprehensible 
% based on their opinions on which feature(s) was most needed 
% at the lowest cost.\par

% In the given weather broadcast video, 
% there were 4 levels of quality for each of the 5 elements. 
% By default, the video set to the lowest level 
% for all elements before any adjustment occurred. 
% %Based on prior study on the perception of the degradation of the 5 features, we designed the 4 levels to be as linear to user's perception as possible. Below are the 4 levels of the 5 elements listed from the worst to the best quality.

% \begin{enumerate}
%     \item Audio Package Loss Repaired with Silence (package loss rate) \cite{watson1996evaluating}: 20\%, 10\%, 5\%, 0\%
%     \item Video Package Loss (package loss rate) \cite{claypool1999effects}: 20\%, 8\%, 4\%, 0\% (20, 8.3, 3.3, 0) 
%     \item Audio Sampling Rate \cite{oeldorf2012bad, noll1993wideband}: 8kHz, 11kHz, 16kHz, 48kHz
%     \item Video Resolution \cite{oeldorf2012bad, knoche2008low}: 120x90, 168x126, 208x156, 240x180
%     \item Video-audio Synchronization (time video behind audio) \cite{steinmetz1996human}: 240ms, 200ms, 160ms, 0ms (new: 1850, 1615, 1050, 0)
% \end{enumerate}

% In the buy-back task, 
% each level of improvement for one feature costs \$2. 
% It would cost the entire budget of \$30 
% to buy all levels of every feature back. 
% Hence, the option of buying back everything was given to the participants,
% in return, there would be no extra payoff remaining for the participant.\par

% Similar to the first experiment, 
% the money spent on each feature during the buy-back task 
% are considered as the true preference 
% the population had towards the 5 video and audio features. 
% The results from the Likert-scaled surveys and QV survey 
% were then compared to the population's true preference 
% to see how different they were.\par


\subsection{Experiment 2 Results} \label{results-2}
In this section, we first present descriptive statistics of the raw data in the second experiment. Then, we discuss our first analysis on the comparison of relative preference intensity across Likert, QV and Buyback. Lastly, we present our second analysis for comparing the relative preference order across the three groups.

\subsubsection{Descriptive Statistics of the Raw Data}


\begin{table}
    \centering
    \caption{
        Experiment one's sample demographics statistics aligns closely with 2019 US census across all groups and subgroup.
    }
    \Description[Experiment one's sample demographics statistics aligns closely with 2019 US census across all groups and subgroup.]{
        Experiment one's sample demographics statistics aligns closely with 2019 US census across all groups and subgroup.
    }
    \label{table:demo_exp1}
    \begin{tabular}{|c|cccc|c|} 
    \hline
     & \begin{tabular}[c]{@{}c@{}}Likert\\Perc \end{tabular} & \begin{tabular}[c]{@{}c@{}}QV100\\Perc \end{tabular} & \begin{tabular}[c]{@{}c@{}}BuyBack\\Perc \end{tabular} & \begin{tabular}[c]{@{}c@{}}Total\\Perc \end{tabular} & \begin{tabular}[c]{@{}c@{}}Census\\Perc* \end{tabular} \\ 
    \hline
    No High School & 11.32\% & 6.67\% & 7.84\% & 8.72\% & 10.22\% \\
    High School & 28.30\% & 33.33\% & 29.41\% & 30.20\% & 27.73\% \\
    Some College  Associate & 24.53\% & 24.44\% & 27.45\% & 25.50\% & 33.09\% \\
    Bachelor's Degree above & 35.85\% & 35.56\% & 35.29\% & 35.57\% & 33.09\% \\ 
    \hline
    18 - 24 & 7.55\% & 11.11\% & 11.76\% & 10.07\% & 13.65\% \\
    25 - 39 & 33.96\% & 31.11\% & 33.33\% & 32.89\% & 30.74\% \\
    40 - 54 & 26.42\% & 26.67\% & 31.37\% & 28.19\% & 28.32\% \\
    55 - 69 & 32.08\% & 31.11\% & 23.53\% & 28.86\% & 27.29\% \\
    \hline
    \end{tabular}
    \vspace{-10px} %temporary
\end{table}


\subsubsection{Relative Preference Intensity Comparison}

\begin{figure}[htpb]
  \centering
  \includegraphics[trim= 1.5in 0in 1.5in 0in, clip, width=\textwidth, keepaspectratio=true]{"content/image/exp2_2_means_traceplot.pdf"}
  \caption{
    The figure shows comparison of the sampling traces of normalized means from Likert, QV and Buyback's Bayesian models. The models model on Likert's votes between [$-2$, $2$], QV's signed-credits between [$-100$, $100$], and Buyback proportions between [$0$, $1$] respectively. Means from the three Bayesian models are normalized to [$-1$, $1$] by the value of their maximum dimension in every trace respectively for this graph. The mapping of the labels on X-axis are: AQ -- Audio Quality, VR -- Video Resolution, AS -- Audio Stability, MS -- Motion Smoothness, AVS -- Audio-video Sync. QV's traces resemble those of Buyback more than Likert. Most of Likert's traces are above 0.8, showing consistent high intensities across aspects. Both QV and Buyback show a greater variation in the relative intensities and have a significant amount of traces dipping down to the range of 0.4 to 0.6 for the least popular aspect.
  }
  \Description[Means traceplots for exp2]{Means traceplots for exp2}
  \label{fig:means_exp2}
\end{figure}

\subsubsection{Relative Preference Order Comparison}
\section{Discussion} \label{discussion}

% Argument where likert is limited in expressiveness. Some aregue incvreasing the nu,ber of likert points.
\section{Conclusion} \label{conclusion}
In this paper, we examined Quadratic Voting, a computational-powered survey method that combines ratings and ranking surveying approaches, in the setting of resource-constrained collective decision-making. Through two randomized controlled experiments and Bayesian analysis, we showed empirically that a QV survey with sufficient voice credits better elicits participants' true preferences than a Likert scale survey, with a medium to high effect size. Furthermore, our study provided the first example of applying QV in a prototypical HCI user study for the CSCW community. While our study demonstrated the potential of QV as a computational tool to facilitate truthful preference elicitation, the goal of this research is \textit{not} to convince decision-makers to replace their Likert scale surveys with QV surveys entirely. Rather, we encourage decision-makers to consider QV as a promising alternative to the Likert scale in resource-constraint scenarios that require information from both ratings and rankings, and researchers to further explore the many open questions we proposed of QV in future work.

\begin{acks}
We thank the voluntary pretest participants who helped us improved the experiment design and system. A big thanks to the experiment participants. Additional thanks to Vinay Koshy, Ziang Xiao, Yu-Chun Grace Yen, Chi-Hsien Eric Yen, Silas Hsu, Sneha Krishna, Meng Huang, and the anonymous reviewers who provided valuable feedback to this work. This work was partially supported by Microsoft, Facebook and Capital One Financial Corporation. 
\end{acks}




% \bibliography{reference}
% \bibliographystyle{ACM-Reference-Format}
\printbibliography

%\appendix

\section{Experiment One Methods}
\subsection{Definitions of the Nine Societal Causes}~\label{cause_def}
In this subsection, we detailed the definitions of the nine societal causes used in the first experiment. We derived these causes from the categorization of charity groups on Amazon Smile \footnote{https://smile.amazon.com/}, to ensure that the nine societal causes covered a broad spectrum of categories. The nine categories were defined as:
\begin{enumerate}[label={},leftmargin=\parindent]
    \item (1) Pets and Animals: Animal Rights, Welfare, and Services; Wildlife Conservation; Zoos and Aquariums
    \item (2) Arts, Culture, Humanities: Libraries, Historical Societies, and Landmark Preservation; Museums; Performing Arts; Public Broadcasting and Media
    \item (3) Education: Early Childhood Programs and Services; Youth Education Programs and Services; Adult Education Programs and Services; Special Education; Education Policy and Reform; Scholarship and Financial Support
    \item (4) Environment: Environmental Protection and Conservation; Botanical Gardens, Parks and Nature Centers
    \item (5) Health: Diseases, Disorders, and Disciplines; Patient and Family Support; Treatment and Prevention Services; Medical Research
    \item (6) Human Services: Children's and Family Services; Youth Development, Shelter, and Crisis Services; Food Banks, Food Pantries, and Food Distribution; Multipurpose Human Service Organizations; Homeless Services; Social Services
    \item (7) International: Development and Relief Services; International Peace, Security, and Affairs; Humanitarian Relief Supplies
    \item (8) Faith and Spiritual: Religious Activities; Religious Media and Broadcasting
    \item (9) Veterans: Wounded Troops Services, Military Social Services, Military Family Support
\end{enumerate}
The participants saw the same definitions when completing the surveys during the study.

\section{Experiment One Results}
\begin{figure}[htpb]
    \centering
    \includegraphics[width=\textwidth, keepaspectratio=true]{content/image/total_contributions_across_conditions.pdf}
    \caption{
       Distributions of the total amount donated by participants across four surveying methods. This figure also included participants that did not donate any amount, whom we excluded from our analysis. We saw two distributions, one centered by $\$9-12$ and the other centered by $\$33$ to $\$35$. We also see more Likert participants donate almost most of their donation quota compared to the QV Groups.
    }
    \Description[Distributions of Total Donation Amounts across Groups for experiment one]{ Distributions of the total amount donated by participants across four surveying methods. This figure also included participants that did not donate any amount. We see two distributions, one centered by $\$9-12$ and the other centered by $\$33$ to $\$35$. We also see more Likert participants donate almost most of their donation quota compared to the QV Groups.}
    \label{fig:total_don_exp1}
\end{figure}

\subsection{Total Donation Amount}~\label{total_donation}
Figure \ref{fig:total_don_exp1} also demonstrates two clusters for the total donation amount. The first cluster centered around $\$9-12$ with the majority in the range of $\$5-20$. This group of people, making up about $60\%$ of the entire sample, donated part of the lottery winning amount but still kept a significant portion for themselves. The other clustered around $\$33$ to $\$35$, suggesting that this group of participants
contributed almost the full amount of the lottery prize. There were approximately $25\%$ of the participants who behaved this way. The total donation amount distribution across four surveying methods were relatively consistent, except that almost twice the proportion of participants in the Likert condition donated almost the full amount compared to the other QV conditions. One possible explanation for the difference is 
the Likert group required less effort compared to that of QV, and participants felt less tempted to earn an extra reward for their time spent in the Likert condition. 

{\change{\begin{figure}[htpb]
    \centering
    \includegraphics[width=\textwidth, keepaspectratio=true]{content/image/contribution_to_org.pdf}
    \caption{
        The distributions of how participants donated within each experiment group for experiment one. This plot removed participants that donated to zero organizations since they were excluded from our original analysis. We see that about 80\% of participants donated to more than two or more organizations across all experiment groups. More than half of all participants donated to more than three organizations.
    }
    \Description[Distributions of how participants donated for experiment one]{The distributions of how participants donated within each experiment group for experiment one. This plot removed participants that donated to zero organizations since they were excluded from our original analysis. We see that across all experiment groups, about 80\% of participants donated to more than two or more organizations. More than half of all participant donated to more than 3 organizations.}
    \label{fig:par_don_exp1}
\end{figure}

\subsection{How Participants Donated}~\label{individual_donation}
To ensure that participants distributed their donation amount, we extracted the information on how each individual donated. \Cref{fig:par_don_exp1} shows the distribution of how the participants contributed to each group. On an aggregated level, only 18.57\% of participants donated to a single charity, and 59.29\% of participants donated to three or more charity.
}}
\subsection{QV Budget Usage}
To understand how participants used their budgets, we examined the percentage of credits consumed. Participants do not need to use up all their budgets in QV. We found no decrease in the median of percentage budget usage -- all around 98\%, as available voice credits increased. QV324 did exhibit a longer tail for percentage budget usage. Participants, in general, used up their budgets as much as they can while they make trade-offs between options, under budget constraints.
\begin{figure}[htpb]
    \centering
    \includegraphics[width=0.43\textwidth, keepaspectratio=true]{content/image/qv_budget_used_distribution.pdf}
    \caption{
      Distribution of Percentage Budget Used in QV36, QV108 and QV324. Percentage budget used is the percentage of voice credits used out of the total voice credits budget available. The medians for all three QVs are around 98\%.
    }
    \Description[Distribution of Percentage Budget Used for experiment one]{Distribution of Percentage Budget Used for experiment one}
    \label{fig:qv_budget_exp1}
\end{figure}


\section{Experiment Two Design}
\subsection{Experiment 2 Flow Chart}
{\change{We present the experiment flow chart for experiment two in \Cref{fig:exp2_flow}. Specifically, \Cref{fig:exp2_store} provides the two interfaces involved in Step 6.

\begin{figure}[htpb]
    \centering
    \includegraphics[width=\textwidth, keepaspectratio=true]{content/image/exp2_flow.pdf}
    \caption{
        We used a within-subjects design for experiment two. We randomly assigned participants into two groups: one group would complete the Likert scale first and then quadratic voting; the other group experienced them in a reversed order.
    }
    \label{fig:exp2_flow}
\end{figure}

\begin{figure}[htpb]
    \centering
    \includegraphics[width=\textwidth, keepaspectratio=true]{content/image/design_task.png}
    \caption{
        The two steps participants encountered when completing Step 6 of the survey. Participants would first need to select which one of the two qualities they would include in their video streaming product. Participants could see real-time changes to the video as they updated the qualities. Once they made their decisions, participants would price each of the elements between \$0 and \$4. Participants would receive a commission worth 10\% of the total price if the buyer accepted their product at their set prices.
    }
    \label{fig:exp2_store}
\end{figure}
}}

\subsection{Definition of the Five Video Elements for Experiment Two}~\label{elem_def}
In experiment two, we designed the research scenario to answer the following question: ``Given a video with unsatisfying quality, under limited bandwidth, how should the bandwidth be allocated to enhance the five video and audio elements, including the motion smoothness \cite{huynh2008temporal}, audio stability \cite{hardman1998successful}, audio quality \cite{knoche2008low}, video resolution \cite{knoche2005can}, and audio-video synchronization \cite{steinmetz1996human}, to obtain an acceptable video streaming experience from the viewers' perspective?'' We selected the five video playback elements based on prior work, and below are their definitions: 

\begin{itemize}
    \item Motion Smoothness \cite{huynh2008temporal, oeldorf2012bad}: refers to how smooth the visuals of the video are. The number of frames transferred from the server to the viewer per second may be impacted under limited bandwidth. Having a low frame rate means that the video feels jerky and slow.
    \item Audio Stability \cite{hardman1998successful}: refers to how smoothly the audio of the video plays. With limited bandwidth, there may be lost audio packets. This creates short intervals of silence during playback, undermining the interpretability of the audio. The higher probability an audio packet may be lost, the more stuttered the audio sounds.
    \item Video Resolution \cite{oeldorf2012bad, knoche2005can}: refers to how sharp the visuals in the video look. With limited bandwidth, one may reduce the video's size by providing a lower resolution. At a lower resolution, the video imagery becomes pixelated and unclear. 
    \item Audio Quality \cite{oeldorf2012bad, noll1993wideband}: refers to how clear and crisp the audio sounds. A lower audio sampling rate needs a lower bandwidth to transmit. With a lower audio sampling rate, the audio sounds more muffled and unclear.
    \item Audio-Video Synchronization \cite{steinmetz1996human}: refers to how well video visuals are matched with the audio playback. Our experiment focused only on the type of asynchronization where the audio plays ahead of the video. Under bandwidth constraint, visuals and audio may be out of sync due to packet loss in visuals or audio.
\end{itemize}



\section{System Design Details}

\begin{figure}
     \centering
     \begin{subfigure}[ht]{0.49\textwidth}
         \centering
         \vspace*{0.065in}
         \includegraphics[width=0.97\textwidth]{content/image/player_loading.png}
         \caption{Video interface when loading}
         \label{fig:video_loading}
     \end{subfigure}
     \hfill
     \begin{subfigure}[ht]{0.49\textwidth}
         \centering
         \includegraphics[width=\textwidth]{content/image/player.png}
         \caption{Video interface when playing}
         \label{fig:video_playing}
     \end{subfigure}
        \caption{To assure video playback consistency, the interface would signal loading when switching video and audio files upon participants changing the toggles. The image to the left shows how the cue was presented. The image to the right shows how the system works under normal conditions.}
        \label{fig:appendix_video_interface}
\end{figure}

\subsection{Experiment two video interface implementation detials}~\label{appx_video_interface}
For this experimental setup, we used AngularJS and bootstrap for the front-end implementation powered by Flask web framework written in Python. We used MongoDB Atlas to store data and Heroku to serve the system. Survey.js rendered all types of surveys besides the QV interface in our experiment. The experiment source code is publicly available \footnote{https://github.com/a2975667/QV-buyback} and so is the standalone QV interface \footnote{https://github.com/hank0982/QV-app}.

In this experiment, the most challenging component is to design a stable, reliable, and real-time video rendering interface for participants to experience how changes in different video element quality contribute to their overall experience. Since we provided four possible levels of adjustments for each element, there will be $4^5 = 1024$ possible combinations, which is impossible to pre-generate and serve to the participants. Therefore, we need a real-time rendering video playback system in our experiment. 

To the best of our knowledge, no video player supports real-time rendering of different video qualities ,and that there is little work that degrades video playback purposely. Thus, to achieve our experiment goal, we need to implement our own video playback system \Cref{fig:exp2_playground}. We broke the video clip into two parts: (1) a video without audio and (2) audio. In our final experiment, we pre-generated $4^2 \time 2 = 32$ files. $4^2 =16$ of which are different levels of video quality ($N=4$) and frame rates ($N=4$) while the other $4^2$ versions are  varying levels of audio quality ($N=4$) and audio stability ($N=4$). We decided not to use the server nor the browser to render the video quality or frame rates in real-time to control what the participants see even in areas of lower internet bandwidth or when there is congestion on the server. We pre-generated the audio files because of similar reasons and pre-generation made sure the locations of packet loss were consistent across participants. FFMPEG was used to generate all 16 files. Video files were first decoded and encoded at the desired resolution, bitrate and framerate. Audio files were first decoded and and encoded at the designated bit rate. We simulated audio stability by randomly losing 40 milliseconds of packets according to the probability listed in \ref{exp2-hci}.

Once these files were pre-generated, even the most distinct environment could see the same video and audio files. Understanding that network environments might delay the transmission of video and audio files, we implemented a spinning wheel in the interface to signal while the files are still loading (\Cref{fig:video_loading}). Once the client received the correct combination of files, it will play both files simultaneously according to the corresponding time anchor (\Cref{fig:video_playing}). A front-end JavaScript determined this time anchor, simulating the video-audio synchronization levels by playing the video and audio files from different start times. 

\subsection{QV interface iterations}~\label{appx_qv_interface}
The current QV interface was designed over multiple iterations. The goal of the interface is to assist participants' voting process using visual information to reduce their cognitive load. \Cref{fig:appendix_qv_interface} portraits the draft QV interface with the current design used in our experiments. Both interfaces featured a voting panel that contained a list of options to vote on. To the left of each option, participants can use the plus and minus buttons to vote for or against an option. Buttons for an item were automatically disabled if the number of voice credits remaining did not permit an additional vote for that item. 

The major difference lies in the information presented to the participants. In the new interface, we provided a bar with the proportion of voice credits contributed to that option rather than simple text under each option. Comparing the two interfaces, the new interface also allowed more description to be present under each of the options. In addition, we floated the summary panel at the bottom of the page at any time to ensure visibility, which provided information on the number of voice credits the participants have and have not used out of the total budget. Finally, we also changed the color scheme of the interface for accessibility reasons.

\begin{figure}
     \centering
     \begin{subfigure}[ht]{0.49\textwidth}
         \centering
         \includegraphics[width=\textwidth]{content/image/old_qv.png}
         \caption{The initial QV interface}
         \label{fig:old_qv_interface}
     \end{subfigure}
     \hfill
     \begin{subfigure}[ht]{0.49\textwidth}
         \centering
         \includegraphics[width=\textwidth]{content/image/new_qv.png}
         \caption{The current QV interface}
         \label{fig:new_qv_interface}
     \end{subfigure}
        \caption{The QV interface was redesigned multiple times. The image on the left showcased the same QV playground in an early iteration of the interface design. The new design provided much more information to reduce participant's cognitive loads.}
        \label{fig:appendix_qv_interface}
\end{figure}

\end{document}
\endinput

