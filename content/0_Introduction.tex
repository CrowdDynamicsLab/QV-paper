% To-dos
% Finish contribution in intro
% related work -- smooth language [pending approval, need another pass -- TC]
% check if RQ is updated in the paper
% -- update traceplot in exp2 -- [Complete]
% align experimental flow format in exp1 and exp2
% update exp2 motivation
% update limitations (move 7.3 to limitations and future work)
% address feedback on discussion
% update conclusion and abstract
% decide title
% complete appendix (search for keyword "appendix") [in process -- TC]
% supplementary materials
% change mentions of "experiment 1 / 2" to "experiment one / two"
% grammar check; final read through

% Finished

\section{Introduction}
Decision makers often utilize surveys to aggregate opinions from the crowd to inform decisions. Understanding a crowd's preferences and prioritization across a set of options under resource constraints is a common use case of surveys, such as public polling to allocate government budget \cite{pew_spending}, choosing the best designs in a human-computer interaction (HCI) interface \cite{ledo2018evaluation}, and prioritizing what products to stock in grocery stores \cite{nielsen}. To make resource-constrained decisions, decision makers ask the crowd two questions: (a) \textit{how much} do you prefer each option, and (b) how would you prioritize (i.e., rank) the options? 

Ratings and rankings are two primary categories of survey methodology; they each focuses on answering one of the two questions, but not both \cite{moors2016two}. Ratings, such as Likert scale \cite{likert1932technique}, elicits attitude intensity to answer question (a), while rankings, such as ranking scale \cite{starch1923methods}, asks about question (b) \cite{moors2016two}. These limitations motivated us to examine an alternative that combines features from ratings and rankings, Quadratic Voting (QV), a relatively new voting mechanism proposed by \textcite{posner2018radical}. In this paper, we empirically tested how well QV elicits true preferences in resource-constrained collective decision making processes compared to Likert scale, the current norm in these processes \cite{moors2016two}.

% In order to find truthful answers, researchers have developed various forms of surveys to elicit \textit{truthful} self-reported preferences \cite{stone1999science}. 
Throughout the past century, researchers developed various forms of surveys in an effort to elicit true preferences via self-reporting. Rating and ranking-based surveys are the two major categories \cite{moors2016two}, each with their strengths and weaknesses. 

When asking opinion-related questions in a rating-based survey, survey takers indicate their level of agreement, satisfaction, or importance along a scale of intensity \cite{moors2016two}. Likert scale is the most popular rating-based technique that elicits the \textit{strength} of a survey taker's preferences \cite{likert1932technique}. Rating-based surveys are easy to understand and administer; therefore, Likert scale is the current norm in many survey settings, including resource-constrained collective decision making. For example, Pew Research Center administered annual surveys with a 3-point Likert-scale to understand how participants would make up the budget for the federal government \cite{pew_spending}. In a resource-constrained scenario, however, Likert scale may not accurately elicit how participants \textit{prioritize} the options, since the statement for each option is independent \cite{alwin1985measurement} and Likert survey does not safeguard against exaggerated or unrealistic preferences \cite{araujo2017much, vavreck2007exaggerated}. In the case of federal government budget survey by Pew Research Center \cite{pew_spending}, participants can freely select ``increase spending" for all government program areas without considering its feasibility under limited government budget.

On the contrary, ranking-based surveys focus on understanding how participants prioritize the options \cite{moors2016two}, naturally incorporating the notion of resource constraints. For example, in participatory budgeting, a resource-constrained collective decision making scenario, participants use ranked voting to rank a list of options in the order of their relative preferences \cite{benade2020preference}. However, ranking-based surveys cannot elicit the strength of a participant's preferences, i.e., how much more a participant values option A over option B. In addition, performing statistical analysis on rankings data is challenging \cite{alwin1985measurement}.

Researchers have been debating between rankings versus ratings for decades due to their respective limitations. Given the ubiquitous computing devices today, we ask: is there a computational-powered survey methodology that elicits both the rankings \textit{and} ratings of a set of options accurately in a resource-constrained collective decision making process? In this study, we investigate Quadratic Voting (QV) \cite{posner2018radical} as a potential answer to this question. 

Participants rank and rate the options in QV at the same time. In QV, expressing an opinion is not free -- each participant receives a fixed budget of voice credits, i.e., the ``currency" in QV, and pays a quadratic cost for the votes they cast. More formally, casting $n$ votes for or against an option costs $n^2$ voice credits. Participants vary the number of votes to represent the strength of their preference, i.e., they \textit{rate} an option with the number of votes. But as the number of votes on an option increases, the marginal cost for each extra vote increases as well; in other words, each additional vote becomes more expensive. Participants may cast as many votes as they choose on any option, as long as the total costs for the votes do not exceed the fixed given budget of voice credits. The limited budget and increasing marginal cost encourage participants to make trade-offs among options, equivalent to \textit{ranking} them. \textcite{posner2018radical} argued that results from this mechanism are more optimal than those from one-person-one-vote systems, such as majority ruling \cite{sep-voting-methods},  because QV minimizes welfare loss. Prior work further demonstrated that QV outcomes approximated a normal distribution, as opposed to the polarized responses from Likert scale survey results on highly-debated topics \cite{quarfoot2017quadratic}. However, to the best of our knowledge, no one has empirically explored whether QV processes elicit more accurate self-reported preferences.

In our study, we examined QV's ability to elicit true preferences in resource-constrained collective decision making processes. Given that Likert scale is the most widely adopted survey instrument to date \cite{moors2016two}, to interrogate our findings, we compared QV to Likert scale \cite{likert1932technique} in a resource-constrained setting. We identified two distinct types of survey questions in this setting: (1) choosing among $K$ independent options of the same subject matter, and (2) choosing among $K$ dependent options that jointly contribute to the same subject matter. To give an analogy, scenario one is about choosing among ice cream, cakes, and puddings for dessert, while scenario two is understanding what an individual cares more about when choosing an ice cream, the flavor, texture, or ingredients. Therefore, we analyzed them as two separate research questions:


\begin{itemize}[leftmargin=0.6in]
    \item[\textbf{RQ 1}] How well do results from QV align with people's true preferences compared to Likert scale when the survey aims to choose among $K$ independent options of the same subject matter? How do variations in the amount of voice credits given to QV survey takers impact this outcome?

    \item[\textbf{RQ 2}] How well do results from QV align with people's true preferences compared to Likert scale when the survey aims to choose among $K$ dependent options that jointly contribute to the same subject matter? 
\end{itemize}

To answer RQ1, we designed a survey in the context of a public polling about what social causes need more support, a typical scenario to choose among $K$ independent options. To measure participants' true preferences on the same topic, we created an incentive-compatible donation task, in which participants should only donate more for a cause if they truly care more about it. Participants on Mechanical Turk (MTurk) completed either the Likert scale version or the QV version of the survey, and the donation task, in a randomized controlled experiment. Since we were not aware of any prior work that provided guidance on how to determine the budget of voice credits in QV empirically, we asked RQ1(a) and explored three variations of the QV survey with small ($O(K)$), medium ($O(K^{1.5})$), and large ($O(K^2)$) budget in the experiment. We measured the similarity between each individual's survey result and their true preferences as reflected in the donation task, and applied Bayesian analysis to compare the degree of similarity to true preferences in QV and Likert scale surveys.

For the case of choosing among $K$ dependent options that jointly contribute to the same subject matter in RQ2, we used an example of an HCI user study in understanding how users make trade-offs across visual and audio elements in a video streaming experience given limited internet bandwidth. We used QV and Likert scale surveys to obtain participants' self-reported responses, and designed a incentive-compatible product design and pricing task to elicit their truthful willingness-to-pay for each video element. Similar to the first experiment, we conducted the randomized controlled experiment on MTurk and analyzed the results with Bayesian analysis.

\textbf{Contributions}
% accuracy
% wider application domains

This work, with a focus on QV, contributes to the extensive body of work on survey techniques that elicit self-reported preferences in resource-constrained decision making in two ways: showing an improved accuracy of responses with QV compared to the norm, Likert scale, and demonstrating the applicability of QV in fields other than public policy and education research.

First, we empirically showed that QV with medium ($O(K^{1.5})$) to large ($O(K^{2})$) voice credit budget elicits true preferences more accurately than Likert scale when the survey goal is to (1) choose among $K$ independent options of the same subject matter, or (2) choose among $K$ dependent options jointly contributing to the same subject matter, under resource constraints. Different from prior empirical work that compared the characteristics of QV and Likert scale results \cite{quarfoot2017quadratic, naylor2017first}, we focus on their ability in accurate preference elicitation. And unlike prior theoretical work that proved QV's incentive-compatibility \cite{lalley2018quadratic, posner2018radical}, we tested QV empirically and examined its results' alignment with responses from an incentive-compatible task. Based on two carefully-designed randomized controlled experiments, our Bayesian analysis showed that QV results aligned significantly closer to participants' behaviors in an incentive-compatible task compared to Likert scale results with a medium to high effect size. This finding is significant because it shows the potential of QV, a computational-powered alternative survey instrument that combines 

This paper offers three major contributions.
First, we designed two empirical studies and found that QV aligns better with the participant's underlying preferences when surveying people's preferences among $K$ options. The first experiment looked at eliciting societal causes while the other experiment pertains to HCI. Both of these experiments used participant behaviors as a baseline to measure these differences, which we argue is much stronger than the direct comparison to Likert that prior studies lean on. The results allowed us to argue that researchers should harness computers' power to utilize and benefit from deploying QV in studies if the study aims to understand people's preferences among $K$ options.

Second, we demonstrate that QV results depend on the voice credit budget. To the best of our knowledge, little did researchers discussed the effect of the number of voice credits in prior works. In this study, we discovered that participants need a large enough voice credit to obtain an accurate QV result.

The third contribution demonstrates the application of QV in the HCI domain. We are not aware of prior studies in HCI that utilize QV, and we believe that our experiment and software deliverable provides a template to use QV in later research.

These findings are critical to QV applications in the wild. It is important to stress that the goal of this research is \textit{not} to convince a replacement of Likert Surveys. Instead, we believe QV as an additional tool inside the researcher's toolkit to facilitate better decision making. We also think this study opens the door to many future directions, such as investigating interface designs for QV.


% ask participants to express their attitudes using a scale for a given topic or question on a scale of 1 to 5. Researchers would use collected responses to make their decision. This paper focuses on collective decision-making tasks aimed at eliciting group preferences among a set of choices among $K$ options. Questions such as: ``Which interface do users like most?'', ``What are some products that customers are satisfied with?'', and  ``What public policies do constituents feel more important?'' are examples of eliciting group preferences among $K$ options.


% \textit{RQ1.} How do results from QV, Likert survey align with people's incentive-compatible behavior when surveying societal issues? 

% We designed an experiment to compare how participants donate with QV and Likert survey results. We analyzed data using Bayesian analysis and qualitative analysis. We hypothesized that QV results better portray individuals' true beliefs as reflected by their monetary donations.

% \textit{RQ2.} How do results from QV and Likert-scale align with people's incentive-compatible behavior with a choice of HCI research questions? 
% To the best of our knowledge, no HCI studies used QV in their studies. Similar to the first experiment, we compare results from QV and Likert surveys with participant's behaviors in a video-element HCI experiment. We hypothesize that QV more accurately reflects the intensity of individuals' true beliefs. 

% \textit{RQ3.} How do different quantities of voice credits impact the results of QV empirically?
% We did not find studies on how many voice credits a participant should receive for QV in related literature. In conjunction with RQ1, we provide participants with a different number of voice credits. We suspect different outcomes from QV when participants have a different number of voice credits.


% is prone to reduced data quality. 
% Since each statement of Likert survey is independent when answering, \textcite{alwin1985measurement} found participants respond with the first acceptable answer instead of compiling an optimal solution. Another fallacy follows similar logic where participants would express opinions to the extreme, exaggerating one's actual belief \cite{araujo2017much, vavreck2007exaggerated}. This is due to Likert survey's nature of not incorporating the notion of resource constraints in the real word where it actually exists, for example, in the case of allocating budget for government spending. Participants in the example could express having large budgets for all causes, which in reality, in not possible because the budget is limited. Some researchers tried to introduce constraints to rating based surveys, for instance knapsack voting, developed by \textcite{goel2015knapsack}, forces participants to elect agreement only if the sum of the cost for the selected items is less than a given budget.

% For example, ranking scales is a ranking-based technique created by \textcite{starch1923methods} in 1923. Participants ranks a list of options in the order of their preferences in a ranking scale survey. Though researchers believe ranking demonstrated better reflects individuals' accordance, they are harder to statistically analysis and more cognitive demanding to participants \cite{moors2016two}. In other words, people are forced to make distinctions between options for rank-based surveys and applying constraints, i.e., there can only be one item ranked first. When taking budget into considerations, researchers developed value-for-money-ranking which asks participants to rank according to not just how they value the options, but the value-cost ration of the options. The biggest drawback of rank-based methods is giving up the information of \textit{strength} among their standings, i.e., how far apart is the first preference compared with the second? 


% Since early 1980s when investigators start to distribute surveys using internet-connected devices, electronic-mediated surveys merely translate paper-based surveys onto a webpage. Researchers did not make noticeable changes on the \textit{survey mechanisms} that unleashes the power of computation. Given the computing power of computers today and the omnipresence of computers at any individual's hands, can a better surveying mechanism harness computational advantages to obtain more accurate survey results?

% In this paper, we examine an alternative, Quadratic Voting (QV), a relatively new voting mechanism proposed by \textcite{posner2018radical}, to elicit more truthful self-reported responses in resource-constrained collective decision making processes. 

% Distribution of surveys for resource-constrained decision making is pervasive in everyday life and these surveys have been effective at aggregating people's attitudes to form consensus. Think tanks design and deploy large-scale ordinal scale polls \cite{pew} to understand public opinions on government policies given limited government resources. Supermarkets collect individuals' shopping experiences across products because there are limited shelves \cite{nielsen}. 

%% electronic decision making surveys merely transform from paper, but hasn't done much extra with the computational power; 
% Before the 1980s, research surveys required large-scale in-person participation. Slowly, surveys evolved into mail-in, telephone, and, eventually, online formats. In the early 1980s, researchers introduced electronic surveys to individuals as internet-connected devices became accessible. \kk{How did people take these? Not too many people had ISP service in the 80s. Which companies administered them?} \tc{They are limited to international companies and fortune 500 in 1985.} Electronic surveys aimed to reduce research costs, reduce human labor required to conducted surveys, and add additional flexibility in survey design\cite{kiesler1986response}. While the \textit{medium} of how researchers conducted the surveys moved from offline to online, researchers did not make noticeable changes on the \textit{survey mechanisms} that unleashes the power of computation. We argue that electronic-mediated surveys merely translate paper-based surveys onto a webpage. Given the computing power of computers today and the omnipresence of computers at any individual's hands, can a better surveying mechanism harness computational advantages to obtain more accurate survey results?

% QV is an example of utilizing the computational power, and have xxx, xxx, xxx proven benefits in previous work; 

% In this study, , given that quadratic computations are difficult to conduct on paper or verbally due to the required cognitive load.

% we also want to confirm if QV is "accurate" enough in soliciting people's opinions
% Since Likert is the most commonly used form of a survey; we compare QV's accuracy with Likert's; 
% Since surveys are deployed to accurately understand people's attitudes, this paper aims to understand how how well QV survey results align with real world behaviors and beliefs?. To the best of our knowledge, there have not been any empirical studies that investigated to what degree does QV results align with participants' underlying true preferences. 


%Self reported data, that is data supplied by the survey participant is at the core of such surveys across a wide range of disciplines, from medicine to social services \cite{stone1999science}. To address this, researchers, have designed different types of survey mechanisms throughout the past century. In 1923, 
%In another survey approach, pairwise preference, developed by \textcite{scheffe1952analysis} in 1952, participants compared two choices by stating their preferences in one of seven intensity levels. Researchers developed these methods to better elicit individual attitudes under different settings and for different goals. Many of the proposed methods, however, fail in capturing authentic signals due to self-reporting bias or <define it>.
% how to use a survey and when to use it is also important
%













%%----------------------------%%




% understanding how to apply QV in real-world scenarios, what challenges QV might face, and design decisions that can influence QV participants.

% \textbf{Design Implication}
% TODO. Talk about interface, future work and insights.


% Our first question focused on whether QV consistently produces a different result from the Likert survey results in an HCI study. Although usually applied to surveys in the social sciences, we hypothesize that QV could produce a cleaner and more accurate result than a Likert survey in an alternative setting such as Human-Computer Interaction (HCI). We designed one pretest experiment and one follow-up experiment to verify our hypothesis. In the pretest, we replicated the study developed by Quarfoot et al. \cite{quarfoot2017quadratic} and swapped the content of the survey with an HCI security study by Leon et al. \cite{leon2013matters} to show that QV extracts more fine-grain information compared to Likert. Results from the pretest aligned with the results by Quarfoot et al., suggesting that QV works under scenarios outside of societal-focused settings. 

%%%%%%%% old text from thesis
% Commercial companies deploy online surveys to understand which improvements users would like to see prioritized for a similar constraint. 
% Other surveys can be found in shopping centers to . 
% Data scientists and product teams also use surveys for them to prioritize mission-critical issues that customers face when deploying a product upgrade. 
% All these examples demonstrate how prevalent groups and institutions used surveys to make decisions by gathering consensus from surveying individual's attitudes.
% Our second question regards how QV and Likert-scale align with an individual's preferences expressed via their behavior across societal issues. We carefully redesigned a new experiment involving a larger group of participants to compare the alignment of QV and Likert surveys with a person's actual behavior. In this experiment, participants expressed their attitudes across societal causes and were given the option to donate to charitable organizations associated with each cause. We took a Bayesian approach to analyze the statistical power and effect size of how closely the Likert Survey and QV align with one's donation distribution. The results found statistical evidence that QV outperforms Likert if there are enough voice credits. Our third question looked at whether a different number of voice credit budget impacts people's voting behavior in QV. We used both experiments, the pretest experiment, and the formal experiment to test this idea. We found through qualitative and quantitative analysis that QV surveys require a significant amount of voice credits. We describe these research question more thoroughly in Chapter \ref{RQ}.

% Since then, this surveying method has been fully adopted by researchers and marketers alike. 
% Some researchers attributed this to the survey's ease of use. However, this ease of use does not guarantee the ease of analysis. Researchers today easily misuse Likert surveys by applying the incorrect analysis methods \cite{bishop2015use} or misinterpreting the analysis results \cite{jamieson2004likert, pell2005use}, leading to doubtful findings. Examples of such includes calculating the mean and standard deviations to interpret the data collected through a Likert survey. Even when applied correctly, there is usually little justification for using Likert surveys over other research tools in their research. 

% and how well it aligns with other surveying methods such as Likert survey. Further, we did not find any empirical study examining whether a different amount of voice credits impacts QV's result.
% Researchers have compared the Likert survey with QV from empirical and theoretical perspectives \cite{quarfoot2017quadratic, naylor2017first}. Cavaille et al. argued that QV outperforms Likert surveys among a set of political and economic issues \cite{cavaille2018towards}. 

% Surveys have been a useful tool at aggregating 
% Research agencies, industry labs, or independent researchers often hope to understand how to allocate resources better or to address people's preferences more accurately. 
% Agencies collect these surveys of huge crowds as a form of collective decision making. 


%%%%%%% old text from QV draft
% Likert survey is one of the most widely used methods to obtain the participant's opinion in the realm of human-computer interaction. Survey participants would express a rating across a series of measurements --- \textit{Very agree to very disagree} or \textit{On a scale of 1 to 5} --- for a listed statement. Very often, these opinions help researchers or decision-makers uncover consensus  across a group of people. However, there had been findings of how researchers can easily misuse Likert surveys either applying incorrect analysis methods  \cite{bishop2015use} or misinterpreting the analysis results \cite{jamieson2004likert, pell2005use} leading to questionable findings.In addition,  many research papers do not explain the rational behind the use of  Likert surveys. In a community that adopted Likert surveys almost as the defacto standard, we ask a fundamental question: ``Is Likert-scale survey the ideal method to measure collective attitudes for decision making?''
% Research agencies, industry labs or independent researchers often want to understand how to better allocate resources or what to improve upon. This is an example of eliciting user preferences  among $K$ options. These opinions are collected among huge crowds, as a form of collective decision making. For example,  ordinal scale polls were designed to understand public opinions on government policy \cite{pew} because there is limited funding. Companies deploy online surveys  to understand how product users  feel about the features and services that needs further improvements because companies have limited time  to develop the next release. Physical surveys can be found  in shopping centers to collect an individual's experiences for products on the shelf because there are limited shelves. All these examples demonstrated how surveys are often tied to  making decisions  by gathering consensus from surveying individual's attitudes. 
% In this study, we look at an alternative method called Quadratic Voting (QV). Published in 2015, \textcite{posner2018radical} proposed Quadratic voting as a voting mechanism with approximate Pareto efficiency.Under this voting mechanism,voters were initially given a fixed amount of voice credits (VC).With the credits, individuals can purchase any number of votes to support any of the statementslisted on the ballot. However, the cost of each vote increases quadratically when voted toward the same option.The authors proved that this mechanism is more efficient at making a collective decision because it minimizes welfare loss.Since 2015, a few studiescompared Likert surveys with QV empirically and theoretically\cite{quarfoot2017quadratic, naylor2017first}.\textcite{cavaille2018towards} argues that QV reflects one's opinion better compared to Likert-scale surveys when surveying one's opinionamong a set of political and economic issues.Despite these findings,we are not aware of related works thatcompare Likert surveys and QVwith participants' underlying true preferences.Therefore, it is unclear whether or notand in what degreedoes QV results align with participants' behaviors.In addition, no current work, to the best of our knowledge,deployed QV in the area of HCI.
%\end{enumerate}
% To answer these research questions,
% we designed two experiments.
% The first experiment,
% designed to answer RQ1 and RQ3,
% is a between-subject study
% where participants express their attitudes
% among a set of societal causes using 
% QV and Likert surveys
% and then donate
% to organizations relevant to these organizations.
% The second experiment 
% created an HCI study environment,
% aimed to answer RQ2,
% where participants were asked about 
% opinions among different video elements
% and their opinions using QV and Likert surveys.
% Our results showed that both experiments support
% QV in providing a clean and efficient way
% compared to Likert surveys
% at eliciting participant's true preferences.
