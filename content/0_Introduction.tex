\section{Introduction}
There are various methods to collect
people's attitudes
Surveys exist 
to collect people's attitudes 
for decision making.
Among all the surveying methods, 
Likert-scale surveys 
are one of the most prominent 
and widely used tools.
Despite its wide adoption,
a Likert-scale survey does not come without its limitations.
While some researchers 
point out how Likert-scale surveys can be misinterpreted \cite{jamieson2004likert, pell2005use},
others pointed out many occasions of incorrect analysis methods \cite{bishop2015use}.
Little research spends time
justifying the use of a Likert scale 
before deploying it in the experiment
, and researchers often view 
Likert-scale surveys as the de facto standard.
This poses a fundamental question: 
``Is Likert-scale survey the ideal method
to measure collective attitudes for decision making?''
In this study, we look at an possible alternative
decision making method: Quadratic voting.

Quadratic voting (QV) is a voting mechanism 
with approximate Pareto efficiency 
developed by Weyl at al. \cite{posner2018radical} in 2015.
Participants in QV were first given a budget.
They can purchase any number of votes 
among the options listed on the ballot 
using their budget.
The cost of the votes would increase 
in quadratics among the same option.
The authors argue that 
this mechanism is more efficient 
at making a collective decision 
because it minimizes welfare loss.
In related works, 
researchers compared Likert scale survey with QV 
from empirical and theoretical perspectives\cite{quarfoot2017quadratic, naylor2017first}.
Cavaille et. al argues that 
QV outperforms Likert-scale surveys among a set of political and economic issues \cite{cavaille2018towards}.
To the best of our knowledge, however, 
there had not been empirical studies 
that investigate whether and in what degree
does QV results align with 
participants' underlying true preferences.
There were also no work that we known of,
that tries to deploy QV 
in the area of Human-Computer Interaction (HCI).

We begin by narrowing the fundamental question
to a specific scenario
where the collective decision making aims to
elicit preference.
Decision making often occurs 
because there are limited resources
and priorities had to be made.
When a group of people 
need to consolidate on a consensus,
surveys are deployed 
to collect individual's attitudes.
For example, 
in order for 
research agencies to understand public opinions
on what matters most to their life,
polls were conducted \cite{}.

Another example would be user researchers trying to prioritize functionalities
among a set of issues to deliver to the customers.
In this experiment, we limit our setting in such scenarios,
or in other words, eliciting a preference among $K$ options.

We hypothesis that 
QV is able to demonstrate a cleaner and more efficient manner 
compared to Likert-scale survey
not only in a generic setting
but also applicable for Human-Computer Interaction (HCI).
We designed one pre-test experiment and two experiment to verify our hypothesis.
In the pre-test,
we replicated the experiment designed by Quarfoot et. al \cite{quarfoot2017quadratic}
with HCI security study by \cite{leon2013matters} to show that
QV demonstrated some differences from the likert-scaled responses
aligning with the results by Quarfoot et. al.
Therefore, we designed two additional experiments
and analyze how these differnces align with participant's true behavior.
In the first experiment, 
we conducted a between-subject study
on people's donation behavior
with their attitude using QV and Likert-scaled surveys among a set of societal causes.
In the second experiment,
we capture how participants make trade-offs among different video elements
and their responses on QV and Likert-scaled surveys.

Our work makes X contributions to the research community. 
First, we conducted several thorough experiments 
to understand how QV can supplement Likert-scaled surveys
when conducting ``choosing 1 in K'' experiments.
Second, we show that .... (results from exp1)
Third, we also demonstrated that ... (results from exp2)
Fourth, we see that ... (result from exp 3)
Finally, we provided the source code of our easy to deploy, 
interactive web platform for QV to the community.


\subsection{Research Question}
RQ1. How well do results from the two surveying methods, Likert and Quadratic Voting, align with people’s true believes when eliciting preference in the case of ``1 in K''?

RQ2.

RQ3.


H1. Result from Quadratic Voting aligns with people’s true belief better because: Scarcity induces more careful deliberation in the decision making process; Relative weightis important in decision prediction; Quadratic Voting directly elicits relative weight, while Likert implicitly assumes that people will keep relative weight in mind when responding; The case of ``1 in K'' involves constraints and trade-offs  among options; Quadratic Voting surveying scheme has constraint built-in while Likert does not.

H2.