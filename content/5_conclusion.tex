\section{Conclusion} \label{conclusion}
In this paper, we examined Quadratic Voting, a computational-powered survey method that combines ratings and ranking surveying approaches, in the setting of resource-constrained collective decision-making. Through two randomized controlled experiments and Bayesian analysis, we showed empirically that a QV survey with sufficient voice credits better elicits participants' true preferences than a Likert scale survey, with a medium to high effect size. Furthermore, our study provided the first example of applying QV in a prototypical HCI user study for the CSCW community. While our study demonstrated the potential of QV as a computational tool to facilitate truthful preference elicitation {\change{in online, resource-constrained surveys}}, the goal of this research is \textit{not} to convince decision-makers to replace their Likert scale surveys with QV-based surveys. {\change{Though QV is an unfamiliar survey instrument to the public, requiring more time for respondents to learn and use a computer interface, we've also identified some promising future research directions such as better understanding generalizability of QV, further user studies of QV respondent's perception, and developing interface design for QV. In conclusion, we encourage decision-makers to consider QV as a promising online alternative to the Likert scale in resource-constrained scenarios that require information about both ratings and rankings.}}  


