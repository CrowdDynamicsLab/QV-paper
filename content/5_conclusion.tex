\section{Conclusion} \label{conclusion}
In this experiment, we built upon existing empirical literature
and analyzed how QV aligns with participants' true preferences compared with Likert scaled surveys.
We examined the transferability of QV when applied to the HCI domain, and we demonstrated the impact of different voice credits.


\section{Future Work} \label{future}

\subsection{Behavior when casting votes} % slightly boring
It is important to point out that Despite only occurring once (P1d659), we think the term ``voting'' might motivate participants to consider how a collective decision was made.
This participant mentioned ``[\textellipsis] I thought a lot of people would probably support faith and spiritual ecatagory  so I watned to try to counterbalance that by voting against it.'' The participants are willing to pay the cost by devoting fewer votes for issues they care and vote against specific ballots. This behavior is not possible in a Likert scaled survey. This requires further experiments to understand how people's mental model works when voting under QV. \par 

\subsection{Ease of use}
In addition, we do not have enough understanding of the level of usability of QV from two aspects.
First, we are not sure how easy it is for participants to understand QV.
In our experiment, participants are far more familiar with Likert-scaled surveys compared to QV, placing a lighter cognitive load.
We acknowledge that participants need additional time and effort to understand and utilize QV throughout our experiment, yet, it is unclear if education level and prior experiences impact participants' understanding of QV and their final choices.
Second, we do not know if there exists diminished effectiveness of QV if there are large amounts of options, meaning selecting a few over hundreds of options.

\subsection{Interface and medium}
Finally, one crucial unanswered question lies in the medium and interface that QV should be deployed.
It is almost certain that QV will be hard to present on paper since it requires complex calculations. 
Besides, we did not examine if there are differences when voting on a laptop or a handheld device.
We also had not uncovered whether different interface designs, visualization, and additional information impact the performance of QV.

