\section{Conclusion} \label{conclusion}
In this paper, we examined Quadratic Voting, a computational-powered survey method that combines ratings and ranking surveying approaches, in the setting of resource-constrained collective decision-making. Through two randomized controlled experiments and Bayesian analysis, we showed empirically that a QV survey with sufficient voice credits better elicits participants' true preferences than a Likert scale survey, with a medium to high effect size. Furthermore, our study provided the first example of applying QV in a prototypical HCI user study for the CSCW community. While our study demonstrated the potential of QV as a computational tool to facilitate truthful preference elicitation, the goal of this research is \textit{not} to convince decision-makers to replace their Likert scale surveys with QV surveys entirely. Rather, we encourage decision-makers to consider QV as a promising alternative to the Likert scale in resource-constraint scenarios that require information from both ratings and rankings, and researchers to further explore the many open questions we proposed of QV in future work.
