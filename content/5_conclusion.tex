\section{Conclusion} \label{conclusion}
In this paper, we examined Quadratic Voting, a computational-powered survey method that combines ratings and ranking surveying approaches, in the setting of resource-constrained collective decision-making. Through two randomized controlled experiments and Bayesian analysis, we showed empirically that a QV survey with sufficient voice credits better elicits participants' true preferences than a Likert scale survey, with a medium to high effect size. Furthermore, our study provided the first example of applying QV in a prototypical HCI user study for the CSCW community. While our study demonstrated the potential of QV as a computational tool to facilitate truthful preference elicitation {\change{in online, resource-constrained surveys}}, the goal of this research is \textit{not} to convince decision-makers to replace their Likert scale surveys with QV-based surveys. {\change{Instead, we hope to spark an interest among the CSCW community to explore a rich set of promising future research directions of QV, such as to compare QV with surveying methods besides the Likert scale survey, better understand the generalizability of QV, and improve the interface design for QV. In conclusion, we encourage decision-makers to consider QV as a promising online alternative to the Likert scale in resource-constrained scenarios where it's beneficial to elicit both the respondents' ratings and rankings preferences.}}  

%further user studies to understand respondents' strategies and perceptions in a QV survey, 


