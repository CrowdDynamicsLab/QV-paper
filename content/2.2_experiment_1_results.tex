\subsection{Experiment 1 Results} \label{results-1}
\subsubsection{Quantitative Analysis Results}

% -- raw data
%     describe the dataset, total # of participants in each group before and after dataset cleaning, demographics of each group;
    
%     donation descriptive statistics: perc of non-zero donation, total donation amount comparison across groups (discuss our test of checking if other factors impact total donation amount -> absolute vs. normalized donation amount), donation distribution across topics between groups
    
%     QV & Likert votes descriptive statistics: votes distribution per topic across groups, budget usage distribution across QV groups
    
% -- data transformation to alignment measurement
%     describe the calculation of cosine similarity angle theta
%     show histogram and other descriptive statistics of the angel data
    
% -- Bayesian formulation
%     why Bayesian
%     the type of analysis question
%     choice of the likelihood function
%     choice of prior distributions
    
% -- Results analysis
%     Tools: PyMC3, MCMC, NUTS
%     Describe fitted values & convergence (trace plots)
%     Describe effect size analysis for comparing Likert and QV
%     Describe effect size analysis for comparing across QVs
    
\subsubsection{Qualitative Analysis Results}
We ask participants to provide a freeform text response on the reason why they made the choices they made
when participants filled out the Likert scale survey or QV survey,
Of all surveys ($N=394$) across both groups, most participants filled out the surveys ($N=331$) based on what they think are the most important issues to them. %84 percent
Besides, a small portion of participants ($N=30$) used their instincts when replying to the survey.
Some participants either think that every aspect is important ($N=7$) or that resources should be equally distributed ($N=7$).

For the participants that said they reply based on what they think is most important to them, 
participants usually perform an ``electing'' over a handful of issues first and then indicate their preference.
% limitation at showing intensity @ Likert
However, we see a few instances in the Likert group where participants would claim one to two options as the most critical causes but electing three or more ``very important''.
For example, P09a47 mentioned, ``I think the environment education and healthcare should be our top priorities right now. Other issues are also important but not as much so.''
while filling out Education, Environment, Health, and Human Services as ``Very important.'' 
Pd80fc mentioned, ``[I] think health and the environment are important'' while putting Pets and Animals, Arts, Culture, Humanities, Environment, and Veteran. as ``Very important.'' 
Though it is unclear why there is this discrepancy, one participant (P9b3ae
) explicitly mentioned ``[\textellipsis] I would answer otherwise, if there were other options, such as not much, or a little bit.''
Similar issues were not present in the QV Group responses.
Participants were able to express more fine-grain preferences.
P1fee1 mentioned, ``I think health and human service are important and beneficial for society.'' while voting six votes for health and five votes for human services. This indicates that despite being ``important'', there is still a difference in weight and shadowed the limitation of Likert scaled survey, where people can be limited by the options they were given.\par

Despite only occurring once (P1d659), we think the term ``voting'' might motivate participants to consider how a collective decision was made.
This participant mentioned ``[\textellipsis] I thought a lot of people would probably support faith and spiritual ecatagory  so I watned to try to counterbalance that by voting against it.'' The participants are willing to pay the cost by devoting fewer votes for issues they care and vote against specific ballots. This behavior is not possible in a Likert scaled survey.\par

To understand how participants in QV voted, we also analyze how participant's responses changed when the number of voice credits changed. 
If the number of voice credits increased, as expected, some participants uniformly increased the number of votes across all nine options, stating that they try to be fair. Also, logically, other participants would devote the additional voice credits to the items of their likes or dislikes. 
It is, however, fascinating to identify how additional credits pushed participants to express more fine-grain preferences.
For example, some participants that had a drastic increase of voice credits, from 36 to 324 voice credits, expressed devoting some options that originally had zero votes. 
P2d9da stated, ``Because now that I have a lot more credits, I felt that I could vote on more issues that mean something to me.'' The participant initially only voted for Environment; however, with 324 votes, the participants voted for all but Faith and Spiritual. This supports our quantitative finding that a limited amount of voice credits suppressed the performance of QV. Participants also reported being freer and submitted more fine-grain opinions. As one participant (P54f23) responded: ``The greater voice quantity allowed me to vary the differences in choices'' more and similarly Pcc4aa reported, ``with more credits i can show what i really like.''

On the contrary, participants are forced to downsize their preferences if credits decreased. Many participants voiced their need to make tradeoffs. P9e5e6 said, ``I think I covered the bare basics.'' and Pe37f2 said, ``Less to go around, so had to knuckle down and allocate the most to what I think is most important.'' Again, this means that it is crucial to have enough points if we want to reflect participant's preferences and they're intensity accurately.

One thing to notice is participants scaled their preferences based on the number of total voice credits. Even though voice credits are one single unit and do not carry any weight, when total voice credits increase, the value of each voice credit devalued, vice versa. P24194 stated, ``I had fewer credits so each vote seemed more expensive.''

This set of qualitative analysis support the quantitative finding that the number of voice credits impacts the performance of QV and there is a need to find the best way at deciding \textit{what} the number of voice credits should be.