% This research showed that if the options are homogeneous, a two-point Likert scale yields as high-reliability coefficient as a multi-category system. The research also emphasized that reliability should not be the only metric when deciding which scale is better than the other scale, much more the opposite. It implies that a scale should be chosen for the purpose it aims to serve and the context it is used in. However, in modern research that utilize Likert surveys, there was often little to no discussions describing why the researchers used a specific type of Likert survey. 
% In the previous subsection, we mentioned that Likert survey choices could limit individuals' degree of freedom to express their attitudes. One open question includes investigating if another type of Likert survey influences participants' behavior compared to QV. In addition,
% Besides, showing an individual how they had allocated their votes could also potentially interfere and encourage voters to vote for or against an option
%We showed promising results of QV in our study -- QV elicited true preferences more accurately from participants than a 5-point Likert scale when making a collective decision to choose among $K$ options. Given the popularity of Likert scale surveys in a wide range of disciplines that involve self-reporting, our research asks an intriguing question -- whether there is an alternative survey method that can elicit accurate opinions by leveraging computational power. However, it is critical to reaffirm that the goal of this study is not to claim one survey method should replace another. As mentioned above, there still exist many open questions to understand the nuances in QV. Survey creators should carefully consider the strengths and weaknesses (e.g., higher cost to educate participants, higher cognitive cost for participants) of QV and select the best suiting survey method in their contexts.
% \subsection{Where are QV's limitations}
% Experiment two shows a slightly different and more nuanced result.
% It is important to know that
% the characteristics of the ballot options 
% are different compared to experiment one.
% On this ballot, the options are \textit{multiple aspects}
% of a same question.
% In other words, 
% one requires all options 
% to form the result of the final outcome.
% Let us use an analogy:
% If experiment one asks you
% ``how much do you prefer between
% coke, fries, and burgers'',
% the second experiment asks you
% ``how important do you think
% freshness, taste, and texture 
% of your beef patty are?''
% Notice the options in the first question 
% are independent while 
% those in the second question 
% collectively decide how good a beef patty is.
% From an intensity perspective 
% of the aggregated opinions, 
% qualitatively we can observe that 
% the preference result from QV aligns closer 
% to the true preference than that from Likert. 
% We have yet to examine its statistical significance 
% due to an experiment design limitation. 
% From a preference ranking perspective, 
% neither of the results from the Likert survey 
% and QV survey
% diverged significantly 
% from the incentive-compatible behaviors 
% that the buyback group demonstrated. 
% However, part of the Likert results 
% did diverge significantly from the QV results, 
% since aspects in Likert results clustered 
% in the higher rankings, 
% while aspects in the QV results 
% spread out more across the lower rankings. 
% Such finding echoes part of the conclusion 
% in the first experiment, 
% where QV is able to show finer-grained preference results 
% than Likert.
% In conclusion,
% we can conclude that QV with a sufficient budget outperforms Likert surveys when the survey aims to elicit preferences in a 1 in $K$ setting.
% There is not enough statistical evidence yet to conclude whether QV aligns to true preference better than Likert if the survey options are multiple aspects of the same subject. Nevertheless, both experiments demonstrated that QV elicits finer grain preferences from people than Likert in both cases, which is an advantage survey designer could leverage to gain more in-depth understanding of people's opinions.
% \subsection{Design Implications}
% Given these discussions, 
% we propose the following design implications. 
% First, QV provides more fine-grain results,
% including the preference and the level of intensity,
% a group has on a particular topic,
% when deciding among one in $K$ options.
% In the CSCW community,
% we believe this can be applied to 
% many collective decision-making processes.
% Form electing a great project among many,
% to redistributing limited resources among those in need,
% QV serves as an alternative tool 
% than traditional Likert surveys.