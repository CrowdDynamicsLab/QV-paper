\section{Discussion} \label{discussion}

\subsection{QV aligning better than Likert}
The goal of this study is to demonstrate empirically that
QV aligns closer to people's true preferences when compared to Likert scale surveys.
We can likely attribute this result as Likert scaled surveys are bounded by an ordinal scale that the survey conductor designed. 
We see a few instances in the Likert group where participants would claim one or two options as the most critical social causes but electing ``very important'' for three or more options as their attitude.
For example, P09a47 mentioned, ``I think the environment education and healthcare should be our top priorities right now. Other issues are also important but not as much so.''
However, the participants filled out Education, Environment, Health, and Human Services as ``Very important.'' 
Pd80fc mentioned, ``[I] think health and the environment are important'' while putting Pets and Animals, Arts, Culture, Humanities, Environment, and Veteran. as ``Very important.'' 
This echoed the quantitative results that there might exist different levels of ``Very important'' in participant's minds but was not able to express them expressively in the Likert scale survey.
One participant (P9b3ae) explicitly mentioned ``[\textellipsis] I would answer otherwise, if there were other options, such as not much, or a little bit.''
Similar issues were not present in the QV Group responses.
Participants in the QV group were able to express more fine-grain preferences.
P1fee1 mentioned, ``I think health and human service are important and beneficial for society.'' while voting six votes for health and five votes for human services. This indicates that despite being ``important'', there is still a difference in weight and shadowed the limitation of Likert scaled survey, where people can be limited by the options they were given.\par

One could argue that it is the design of Likert Scaled surveys which bounded the results, however,
a five or seven-point Likert Scale survey can be viewed as the de facto method at collecting user attitudes.
There is usually little, if not no, discussion in research studies where it justifies the use of a five or seven-point Likert Scale in the survey.
In fact, one related work discussed the use of different points in Likert scale surveys dated back to 1965 \cite{komorita1965number}. 
This research showed that if the options are homogeneous, a two-point Likert scale yields as high-reliability coefficient as a multi-category system.
The research also emphasized that reliability should not be the only metric when deciding which scale is better than the other scale, much more the opposite.
It implies that a scale should be chosen for the purpose it aims to serve and the context it is used in.
Similarly, it is important to emphasize that our claim is not to veto the use of Likert scale surveys, but to pose the alternative method, QV, that it aligns much closer to participants true preference and provides much more information when making a collective decision.\par

\subsection{Impact of QV voice credits}
In experiment one, we also confirmed our hypothesis that the number of voice credits does impact the results. 
In fact, given a fixed set of voice credits, there exists is a finite set of ways one could allocate their votes. 
Different from Likert scale surveys, the votes change according to the number of options present.
Yet, the performance of QV does not come through before reaching an excessive amount of voice credits, as we see that QV with 36 voice credits unperformed QV with 108 and 324 voice credits.

From the change in participant's responses as the number of voice credits changed, we found supporting evidence that participants require enough votes to demonstrate their preferences.
For example, some participants that had a drastic increase of voice credits, from 36 to 324 voice credits, expressed devoting some options that originally had zero votes. 
P2d9da stated, ``Because now that I have a lot more credits, I felt that I could vote on more issues that mean something to me.'' The participant initially only voted for Environment; however, with 324 votes, the participants voted for all but Faith and Spiritual. This supports our quantitative finding that a limited amount of voice credits suppressed the performance of QV. Participants also reported being freer and submitted more fine-grain opinions. As one participant (P54f23) responded: ``The greater voice quantity allowed me to vary the differences in choices'' more and similarly Pcc4aa reported, ``with more credits i can show what i really like.''
This reflects that additional credits pushed participants to express more fine-grain preferences.
% If the number of voice credits increased, as expected, some participants uniformly increased the number of votes across all nine options, stating that they try to be fair. Also, logically, other participants would devote the additional voice credits to the items of their likes or dislikes. 

On the contrary, participants are forced to downsize their preferences if credits decreased. Many participants voiced their need to make tradeoffs. P9e5e6 said, ``I think I covered the bare basics.'' and Pe37f2 said, ``Less to go around, so had to knuckle down and allocate the most to what I think is most important.'' 
Again, this means that it is crucial to have enough points if we want to reflect participant's preferences and they're intensity accurately.
The question of how to identify \textit{what} number of voice credits to use is still unknown.

\subsection{Making tradeoffs in QV}
One of the biggest advantages that QV poses is limiting the degree of freedom when participants fill out the survey.
A Likert scale survey is hard to translate the idea of ``scarceness'' into a survey because survey respondents can put extreme values for all options. 
QV forced participants to think across options. 
Even though voice credits are one single unit and do not carry any weight, when total voice credits increase, the value of each voice credit devalued, vice versa. P24194 stated, ``I had fewer credits, so each vote seemed more expensive.''
Participants are now forced to think, weighing in an underlying cost, when voting options, pushing it to align better with their donation behaviors.
This finding aligns with the physiological finding by \textcite{Shah2015a}
The researchers discovered that people are more rational when faced with scarcity and making tradeoffs become more salient in this situation.

\subsection{Potential QV use cases in HCI}
TODO

